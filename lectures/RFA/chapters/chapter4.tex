\chapter{Lebesgue integral}

\begin{itemize}
    \item First, we will define the integral for non-negative simple (measurable)
    functions.

    \item Second, we will extend the definition to non-negative measurable functions.
    
    \item Finally, we will define the integral for general measurable functions.
\end{itemize}

\section{Integral of non-negative simple functions}

\begin{fdefinition}

    Let $s: X \to [0, \infty]$ be a measurable and simple function:

    $$s = \sum_{i = 1}^k a_i \cdot \chi_{A_i}$$

    where $a_i \geq 0$ and $A_i \in \M$. Let $E \in \M$. Then, we define the
    \textbf{(Lebesgue) integral} of $s$ over $E$ as:

    $$\int_{E} s \, d\mu = \sum_{i = 1}^k a_i \cdot \mu(A_i \cap E)$$
    
\end{fdefinition}

\begin{fremark}
    There are some remarks:
    \vspace{1em}

    \begin{enumerate}
        \item $s: [a,b] \to [0, \infty)$, $\mu, \mu = \lambda$ (Lebesgue measure)\\
        Then, $\int_{[a,b]} s \, d\mu =$ area under the graph of $s$ in $[a,b]$
        \vspace{1em}

        \item We are already using $0 \cdot \infty = 0$ in the definition. In particular,
        
        $$a_i \cdot \mu(A_i \cap E) = \begin{cases}
            0 & a_i = 0 \\
            \infty & a_i > 0
        \end{cases}
        $$

        if $\mu(A_i \cap E) = \infty$.
        \vspace{1em}

        \item $D \in \M$, then $\chi_D$ is a simple function, and:
        
        $$\int_{E} \chi_D \, d\mu = \mu(D \cap E)$$

        \vspace{1em}

        \item More generally, $s$ simple and measurable, $E \in \M$, then:
        
        $$\int_{E} s \, d\mu = \int_{X} s \cdot \chi_E \, d\mu$$

    \end{enumerate}

\end{fremark}

\vspace{1em}

\begin{fproperties}[Basic properties]
    Let $N, E, F \in \M$, $s_1, s_2: X \to [0, \infty)$ simple and measurable functions. Then:
    \vspace{1em}

    \begin{enumerate}[label=(\roman*)]
        \item If $\mu(N) = 0$, then:
        $$\int_{N} s_1 \, d\mu = 0$$
        \vspace{1em}

        \item If $0 \leq c \leq \infty$, then:
        $$\int_{E} c \cdot s_1 \, d\mu = c \cdot \int_{E} s_1 \, d\mu$$
        \vspace{1em}

        \item $\int_{E} (s_1 + s_2) \, d\mu = \int_{E} s_1 \, d\mu + \int_{E} s_2 \, d\mu$
        \vspace{1em}

        \item If $s_1 \leq s_2$, then:
        $$\int_{E} s_1 \, d\mu \leq \int_{E} s_2 \, d\mu$$
        \vspace{1em}

        \item if $E \subset F$, then:
        $$\int_{E} s_1 \, d\mu \leq \int_{F} s_1 \, d\mu$$
    \end{enumerate}

    The properties $(ii)$ and $(iii)$ are called \textbf{linearity} of the integral.
    The properties $(iv)$ and $(v)$ are called \textbf{monotonicity} of the integral.

\end{fproperties}

\vspace{1em}

\begin{fproposition}
    Let $s: X \to [0, \infty)$ be a simple measurable function. Then, the function:

    $$\phi(E) := \int_{E} s \, d\mu: \M \to [0, \infty]$$

    is a measure on $(X, \M)$.
\end{fproposition}

\begin{proof}
    Let $s = \sum_{i=1}^k a_i \cdot \chi_{A_i}$, $0 \leq a_i \leq \infty$. We have to show that:

    \begin{enumerate}
        \item $\phi: \M \to [0, \infty]$?: Yes, since $s \geq 0$, $\phi(E) \geq 0, \; \forall E \in \M$.
        
        \item $\phi(\emptyset) = 0$?: Yes, since $\int_{\emptyset} s \, d\mu = 0$, as $\mu(\emptyset) = 0$.
        
        \item $\sigma$-additivity?: Let $\{E_n\}_{n \in \N}$ be a sequence of pairwise disjoint sets in $\M$, 
        and $E = \bigcup_{n} E_n$. Then, we have that:

        $$\phi(E) = \int_{E} s \, d\mu = \int_{X} s \cdot \chi_E \, d\mu = \sum_{i=1}^k a_i \cdot \mu(A_i \cap E)$$
        $$= \sum_{i=1}^k a_i \cdot \mu\left(\bigcup_{n} A_i \cap E_n\right)$$

        Since $\mu$ is $\sigma$-additive, we have that:

        $$= \sum_{i=1}^k a_i \sum_{n}  \cdot \mu(A_i \cap E_n)$$
        $$= \sum_{n} \sum_{i=1}^k a_i \cdot \mu(A_i \cap E_n)$$
        $$= \sum_{n} \int_{E_n} s \, d\mu = \sum_{n} \phi(E_n)$$
    \end{enumerate}

\end{proof}

\section{Integral of non-negative measurable functions}

\begin{fdefinition}
    Let $f: X \to [0, \infty]$ be a measurable function, $E \in \M$. Then, we define the 
    \textbf{(Lebesgue) integral} of $f$ over $E$ as:

    $$\int_{E} f \, d\mu = \sup\left\{\int_{E} s \, d\mu: s \text{ simple, measurable and } 0 \leq s \leq f\right\}$$
\end{fdefinition}

\begin{fremark}
    There are some remarks:
    \vspace{1em}

    \begin{enumerate}
        \item If $f$ is simple, then the definition coincides with the previous one.
        \vspace{1em}

        \item $(\N, \power(\N), \mu_{\#})$. Then $f: \N \to [0, \infty]$ is a sequence. Indeed, if we
        name $f_n = f(n)$, then:

        $$\int_{\N} f \, d\mu_{\#} = \sum_{n} f_n$$
        \vspace{1em}

        \item All the basic properties of the integral for simple functions above hold for this
        new definition.
    \end{enumerate}

\end{fremark}

\vspace{1em}

\begin{note}
    The following propositions assume that $(X, \M, \mu)$ is a complete measure space (needed for
    a.e. properties).
\end{note}

\begin{fproposition}[Chebychev's inequality]
    Let $f: X \to [0, \infty]$ be a measurable function, and $0 < c < \infty$. Then:
    
    $$\mu(\{f \geq c\}) \leq \frac{1}{c} \int_{\{f \geq c\}} f \, d\mu \leq \frac{1}{c} \int_{X} f \, d\mu$$

    where $\{f \geq c\} = \{x \in X: f(x) \geq c\}$.

\end{fproposition}

\begin{proof}
    $$\int_{X} f \, d\mu \geq \int_{\{f < c\}} f \, d\mu \geq \int_{\{f < c\}} c \, d\mu = c \cdot \mu(\{f < c\})$$

    Now we just divide by $c$.
\end{proof}

\begin{note}
    We have as a consequence the following lemmas:
\end{note}

\begin{flemma}[Vanishing lemma]
    Let $f: X \to [0, \infty]$ be a measurable function, $E \in \M$:

    $$\int_{E} f \, d\mu = 0 \iff f = 0 \text{ a.e. on } E$$
\end{flemma}

\begin{proof}
    The proof goes as follows:

    \begin{itemize}
        \item[($\Leftarrow$)]: Trivial
        
        \item[($\Rightarrow$)]: We have to show that:
        $$\mu(\{x \in E: f(x) > 0\}) = 0$$

        Let us define $F = \{x: f(x) > 0\} = \bigcup_{n} F_n$, where $F_n = \{x: f(x) \geq 1/n\}$.
        Then, we have that:

        $$F_n \subset F_{n+1} \quad \forall n$$

        so $F_n \uparrow F$. Then, we have that:

        $$\mu(F_n) \to \mu(F)$$

        and:

        $$ 0 \leq \mu(F_n) = \mu(\{f \geq \frac{1}{n}\}) \leq \frac{1}{1/n} \int_{E} f \, d\mu = 0$$

        Then, $\mu(F) = 0$.

    \end{itemize}
\end{proof}

\begin{fremark}
    The vanishing lemma applies to \textbf{every f} once $\mu(E) = 0$, indeed, every property
    is true a.e. on negligible sets. \say{The Lebesgue integral does not see negligible sets}.
\end{fremark}

\begin{flemma}
    Let $f: X \to [0, \infty]$ be a measurable function. Then:

    $$\int_X f \, d\mu < \infty \implies \mu(\{f = \infty\}) = 0$$
\end{flemma}

\begin{proof}
    Exercise. (Hint: $\{f = \infty\} = \bigcap_n \{f \geq n\}$)
\end{proof}

\begin{ftheorem}[Monotone Convergence Theorem (MCT)]
    Let $\{f_n\}_{n \in \N}$ be a sequence of measurable functions $f_n: X \to [0, \infty]$.
    Assume that:
    \vspace{1em}
    \begin{enumerate}[label=(\roman*)]
        \item $f_n \leq f_{n+1} \quad \forall n$
        \vspace{1em}
        \item $\lim_{n \to \infty} f_n(x) = f(x) \quad \text{for } a.e. x \in X$
        \vspace{1em}
    \end{enumerate}

    Then, we have that:

    $$\lim_{n \to \infty} \int_{X} f_n \, d\mu = \int_{X} f \, d\mu$$

\end{ftheorem}

\begin{fremark}
    All assumptions are essential
\end{fremark}

\begin{proof}
    The proof goes as follows:\\

    \textbf{\underline{Part 1:}}\\

    Assume that assumptions (i) and (ii) hold $\forall x \in X$. We have some basic facts:

    \begin{itemize}
        \item $f(x) = \lim_{n \to \infty} f_n(x) \implies f(x) \geq 0$ and measurable.
        
        \item $\int_{X} f_n \, d\mu \leq \int_{X} f_{n+1} \, d\mu$. Then, if we define:
        
        $$\alpha_n = \int_{X} f_n \, d\mu, \quad \alpha = \lim_{n \to \infty} \alpha_n$$

        we have that $\alpha_n \leq \alpha_{n+1}$, so $\alpha_n \uparrow \alpha$. Moreover,
        we have that:

        $$f_n(x) \leq f(x) \implies \int_{X} f_n \, d\mu \leq \int_{X} f \, d\mu$$
        $$\implies \alpha \leq \int_{X} f \, d\mu$$
    \end{itemize}

    So, to complete part 1, we have to show that $\alpha \geq \int_{X} f \, d\mu$.\\

    We use the definition of $\int_X f \, d\mu$:

    Take any $s: X \to [0, \infty)$ simple, measurable and $0 \leq s \leq f$. Take also $0 \leq c < 1$.
    Then, we have that:

    $$0 < c \cdot s \leq f$$

    Take $f_n(x) \uparrow f(x) \; \forall x \in X$. Consider $E_n = \{x \in X: f_n(x) \geq c \cdot s(x)\} \in \M$.
    Then, we have that:

    \begin{enumerate}[label=(\alph*)]
        \item $E_n \subset E_{n+1}$: indeed, $x \in E_n \iff f_n(x) \geq c \cdot s(x) \implies f_{n+1}(x) \geq c \cdot s(x) \iff x \in E_{n+1}$
        
        \item $\bigcup_n E_n = X$: indeed, either $f(x) = 0 \implies x \in E_n \; \forall n$ or $f(x) > 0$ and 
        $c \cdot s(x) < f(x)$. Since $f_n(x) \uparrow f(x)$, we have that $\exists N_0$ s.t. $f_{N_0}(x) \geq c \cdot s(x)$.
        Then $x \in E_{N_0}$.
    \end{enumerate}

    Then, we have that:

    $$\alpha \geq \alpha_n = \int_{X} f_n \, d\mu \geq \int_{E_n} c \cdot s \, d\mu = c \cdot \int_{E_n} s \, d\mu$$
    $$= c \cdot \phi(E_n)$$

    (where $\phi(E) = \int_{E} s \, d\mu$ is a measure). Then, notice that $E_n \uparrow X$, so $\phi(E_n) \to \phi(X)$.

    Then, we have that:

    $$\alpha \geq c \cdot \phi(X) = c \cdot \int_{X} s \, d\mu$$

    Then, $\forall c < 1$, $\forall s$:

    $$\alpha \geq c \int_{X} s \, d\mu$$

    If we take the limit $c \to 1$, we have that $\alpha \geq \int_{X} s \, d\mu$. And if we take the supremum
    over all $s$, we have that:

    $$\alpha \geq \int_{X} f \, d\mu$$

    \textbf{\underline{Part 2:}}\\

    Now, we have to show that the result holds for $a.e. \; x \in X$. Define 
    $$F = \{x \in X: \text{either } (i) \text{ or } (ii) \text{ fails}\}$$

    Then we have that $\mu(F) = 0$, and $E = X \setminus F$. For any $g$ (non-negative, measurable), 
    we have that:

    $$g - \chi_E \cdot g = 0 \quad \text{a.e. on } X$$

    Then, we use the vanishing lemma to show that:

    $$\int_{X} (g - \chi_E \cdot g) \, d\mu = 0$$
    $$\iff \int_{X} g \, d\mu = \int_{E} g \, d\mu$$

    Finally:

    $$\int_X f \, d\mu = \int_{E} f \, d\mu = \lim_{n \to \infty} \int_{E} f_n \, d\mu = \lim_{n \to \infty} \int_{X} f_n \, d\mu$$

\end{proof}


\begin{fremark}
    Note that we now have 2 ways to compute the integral of a non-negative measurable function:

    \begin{itemize}
        \item $\int_{X} f \, d\mu = \sup\left\{\int_{X} s \, d\mu: s \text{ simple, measurable and } 0 \leq s \leq f\right\}$
        \vspace{1em}
        \item $\int_{X} f \, d\mu = \lim_{n \to \infty} \int_{X} f_n \, d\mu$ where $f_n \uparrow f$ simple and measurable functions.
        
    \end{itemize}
\end{fremark}

\vspace{1em}

\begin{fcorollary}[Monotone convergence for series]
    Let $\{f_n\}_{n \in \N}$ be a sequence of measurable functions $f_n: X \to [0, \infty]$.
    Then, we have that:

    $$\int_{X} \sum_{n} f_n \, d\mu = \sum_{n} \int_{X} f_n \, d\mu$$
    
\end{fcorollary}

\vspace{1em}

\begin{fproposition}
    Take $\Phi: X \to [0, \infty]$ measurable, $E \in \M$. Define:

    $$\nu(E) = \int_{E} \Phi \, d\mu$$

    Then, $\nu$ is a measure on $(X, \M)$. Moreover, for $f: X \to [0, \infty]$ measurable:

    $$\int_{X} f \, d\nu = \int_{X} f \cdot \Phi \, d\mu$$

\end{fproposition}

\begin{proof}

    The proof goes as follows:

    \begin{itemize}
        \item $\nu: \M \to [0, \infty]$: Trivial
        
        \item $\nu(\emptyset) = 0$: Trivial
        
        \item $\sigma$-additivity: Let $\{E_n\}_{n \in \N}$ be a sequence of pairwise disjoint sets in $\M$,
        and $E = \bigcup_{n} E_n$. Then, we have that:

        $$\nu(E) = \int_{E} \Phi \, d\mu = \int_{X} \Phi \cdot \chi_E \, d\mu = \sum_{n} \int_{X} \Phi \cdot \chi_{E_n} \, d\mu$$
        $$= \sum_{n} \int_{E_n} \Phi \, d\mu = \sum_{n} \nu(E_n)$$
    \end{itemize}

\end{proof}

\begin{flemma}[Fatou]
    Let $(X, \M, \mu)$ be a complete measure space, and 
    $\{f_n\}_{n \in \N}$ be a sequence of measurable functions. Then:

    $$\int_{X} \liminf_{n} f_n \, d\mu \leq \liminf_{n} \int_{X} f_n \, d\mu$$
\end{flemma}

\begin{proof}
    Recall that:

    $$\liminf_{n} f_n = \lim_{n \to \infty} \left(\inf_{k \geq n} f_k\right)$$
    $$= \sup_{n} \left(\inf_{k \geq n} f_k\right)$$

    Then, we define:

    $$g_n = \inf_{k \geq n} f_k$$

    We have the following properties $\forall n$:

    \begin{itemize}
        \item $g_n$ is measurable.
        
        \item $g_n \geq 0$
        
        \item $g_n \leq g_{n+1}$ 
        
        \item $g_n \leq f_n$
    \end{itemize}

    Then, by the MCT, we have that:

    $$\int_{X} \liminf_{n} f_n \, d\mu = \int_{X} \lim_{n} g_n \, d\mu = \lim_{n} \int_{X} g_n \, d\mu$$
    $$= \liminf_{n} \int_{X} g_n \, d\mu \leq \liminf_{n} \int_{X} f_n \, d\mu$$
\end{proof}

\section{Integral of real-valued measurable functions}

Let $f: X \to \R$ be a measurable function. Then, we can write $f = f^+ - f^-$, where:
$$f^+(x) = \max\{f(x), 0\} \quad f^-(x) = \max\{-f(x), 0\}$$

Notice that $f^+, f^- \geq 0$ are measurable functions. Then, we define:<
$$|f| = f^+ + f^-$$

We also notice that $|f| = f^+ + f^- \geq 0$ is measurable. 

\begin{fdefinition}
    We say $f: X \to \R$ is \textbf{integrable} on $X$ if it is measurable and:
    
    $$\int_{X} |f| \, d\mu < \infty$$

    We define the set of \textbf{integrable functions} as:

    $$\Lcur^1(X, \M, \mu) = \{f: X \to \R: f \text{ is integrable}\}$$

    For $f \in \Lcur^1(X, \M, \mu)$, and $E \in \M$, we define:

    $$\int_{E} f \, d\mu = \int_{E} f^+ \, d\mu - \int_{E} f^- \, d\mu$$

\end{fdefinition}

\begin{fproposition}
    Let $f: X \to \R$ be a measurable function. Then:
    \vspace{1em}
    \begin{enumerate}[label=(\roman*)]
        \item $f \in \Lcur^1 \iff |f| \in \Lcur^1 \iff (f^+ \in \Lcur^1 \text{ and } f^- \in \Lcur^1)$
        \vspace{1em}

        \item (Triangular inequality):
        $$\left|\int_{E} f \, d\mu\right| \leq \int_{E} |f| \, d\mu$$
    \end{enumerate}
\end{fproposition}

\begin{proof}
    The proof goes as follows:
    \begin{enumerate}[label=(\roman*)]
        \item Trivial (but see next remark)
        
        \item We have that:
        $$\left|\int_{E} f \, d\mu\right| = \left|\int_{E} f^+ \, d\mu - \int_{E} f^- \, d\mu\right|$$
        $$\leq \left|\int_{E} f^+ \, d\mu\right| + \left|\int_{E} f^- \, d\mu\right| = \int_{E} f^+ \, d\mu + \int_{E} f^- \, d\mu$$
        $$= \int_{E} f^+ + f^- \, d\mu = \int_{E} |f| \, d\mu$$

        \end{enumerate}
\end{proof}

\begin{fremark}
    In general, it is not true that $|f|$ measurable $\implies f$ measurable. 
    Take $F \subset X$, $F \notin \M$ and:

    $$f(x) = \chi_F(x) - \chi_{X \setminus F}(x)$$

    Then, $|f| = 1$ is measurable, but $f$ is not.
\end{fremark}

\vspace{1em}

\begin{fproposition}
    We propose two properties:
    \vspace{1em}
    \begin{enumerate}[label=(\roman*)]
        \item $\Lcur^1(X, \M, \mu)$ is a (real) vector space.
        \vspace{1em}

        \item The functional 
        $$I(\cdot) := \int_{X} \cdot \, d\mu: \Lcur^1(X, \M, \mu) \to \R$$
        
        is a linear functional.
    \end{enumerate}
\end{fproposition}

\begin{proof}
    The proof sketch goes as follows:\\

    Let $u, v \in \Lcur^1(X, \M, \mu)$, $\alpha, \beta \in \R$. We should
    show that:

    $$\alpha u + \beta v \in \Lcur^1(X, \M, \mu)$$

    since:

    $$|\alpha u + \beta v| \leq |\alpha u| + |\beta v|$$

    Then:

    $$\int_{X} (\alpha u + \beta v) \, d\mu \leq \int_{X} |\alpha u + \beta v| \, d\mu \leq  \int_{X} |\alpha u| \, d\mu + \int_{X} |\beta v| \, d\mu < \infty$$

    since $|\alpha u|, |\beta v| \in \Lcur^1(X, \M, \mu)$.
    Then, we have that $\alpha u + \beta v \in \Lcur^1(X, \M, \mu)$.\\

    For the second property, we have that:

    $$I(\alpha u + \beta v) = \int_{X} (\alpha u + \beta v) \, d\mu = \alpha \int_{X} u \, d\mu + \beta \int_{X} v \, d\mu = \alpha I(u) + \beta I(v)$$
\end{proof}

\begin{fremark}
    All the other basic properties of the integral of non-negative functions
    can be extended to the integral of real-valued functions.
\end{fremark}

\vspace{1em}

\begin{ftheorem}[Vanishing lemma]
    Let $(X, \M, \mu)$ be a complete measure space,
    and $f, g \in \Lcur^1(X, \M, \mu)$. Then:

    $$f = g \text{ a.e.} \iff \int_{X} |f - g| \, d\mu = 0 \iff \int_{E} (f - g) \, d\mu = 0 \; \forall E \in \M$$
    
\end{ftheorem}

\begin{proof}
    The \say{difficult} part of the proof is:

    $$ \int_{E} (f - g) \, d\mu, \quad \forall E \in \M \implies f = g \text{ a.e.}$$

    The proof goes as follows:

    Let $E_1 = \{f \geq g\}$, and $E_2 = X \setminus E_1$. Then, we have that:

    $$0 = \int_{E_1} (f - g) \, d\mu = \int_{E_1}  (f - g)^+ \, d\mu$$
    $$0 = \int_{E_2} (f - g) \, d\mu = -\int_{E_2}  (f - g)^- \, d\mu$$

    Then, we have that:

    $$(f - g)^+ = 0 \text{ and } (f - g)^- = 0 \text{ a.e. on } X$$
\end{proof}

\begin{fremark}
    In particular, for $u \in \Lcur^1$:

    $$\int_{E} u \, d\mu = 0 \; \forall E \in \M \implies u = 0 \text{ a.e.}$$ 

    This is the same as:

    $$\int_{X} u \varphi \, d\mu = 0 \quad \forall \varphi \text{ characteristic function} \implies u = 0 \text{ a.e.}$$

    This can be true also replacing $\varphi$ by \say{something else}. For
    instance, in the case of $u \in \Lcur^1(\R, \Lcur(\R), \lambda)$:

    $$\int_{\R} u \varphi \, d\lambda = 0 \quad \forall \varphi \in V \implies u = 0 \text{ a.e.}$$

    where $V = \{C_0^{\infty}(\R)\}$, or
    $V = \{C_0^0(\R)\}$.

    This is the \say{fundamental lemma of calculus of variations}.

\end{fremark}

\vspace{1em}

\begin{ftheorem}[Dominated convergence theorem (DCT)]
    Let $(X, \M. \mu)$ be a complete measure space and $\{f_n\}_{n \in \N}$ be 
    a sequence of measurable functions $f_n: X \to \R$,
    and $f: X \to \R$. Assume that:
    \vspace{1em}
    \begin{enumerate}[label=(\roman*)]
        \item $|f_n| \leq g$ a.e. on $X$, $\forall n \in \N$, where $g \in \Lcur^1(X, \M, \mu)$
        \vspace{1em}
        \item $\lim_{n \to \infty} f_n(x) = f(x)$ for a.e. $x \in X$
        \vspace{1em}
    \end{enumerate}

    Then, $f \in \Lcur^1(X, \M, \mu)$, and:

    $$\lim_{n \to \infty} \int_{E} |f_n - f| \, d\mu = 0$$

    In particular:

    $$\lim_{n \to \infty} \int_{X} f_n \, d\mu = \int_{X} f \, d\mu$$
    
\end{ftheorem}

\begin{proof}
    First, we have 2 basic facts:

    \begin{enumerate}
        \item $|f_n| \leq g$ a.e. on $X$, $\forall n \in \N \implies f_n \in \Lcur^1(X, \M, \mu)$
        
        \item $|f| \leq g$ a.e. on $X \implies f \in \Lcur^1(X, \M, \mu)$
    \end{enumerate}

    Then, consider the sequence $h_n = 2g - |f_n - f|$. We have that:

    \begin{itemize}
        \item $h_n$ is measurable.
        \item $h_n \leq 2 g$
        \item $h_n \geq 0$. Indeed:
        $$|f_n - f| \leq |f_n| + |f| \leq 2g \implies 2g - |f_n - f| \geq 0$$
    \end{itemize}

    We now apply the Fatou's lemma to the sequence $h_n$:

    $$\int_{X} (\liminf_{n} h_n) \, d\mu \leq \liminf_{n} \int_{X} h_n \, d\mu$$
    $$ = \int_{X} 2g \, d\mu - \limsup_{n} \int_{X} |f_n - f| \, d\mu$$

    Also, notice that:
    $$\liminf_{n} h_n = 2g$$

    Then, we have that:

    $$\int_{X} 2g \, d\mu \leq \int_{X} 2g \, d\mu - \limsup_{n} \int_{X} |f_n - f| \, d\mu$$
    $$\implies \limsup_{n} \int_{X} |f_n - f| \, d\mu \leq 0$$

    Then, we have that:
    $$\limsup_{n} \int_{X} |f_n - f| \, d\mu \geq \liminf_{n} \int_{X} |f_n - f| \, d\mu \geq 0$$

    In the end:

    $$\lim_{n} \int_{X} |f_n - f| \, d\mu = 0$$
\end{proof}

\begin{fremark}
    If $\mu(X) < \infty$, then the constants are integrable. Then, if
    $|f_n(x)| \leq M$ a.e, for some $M \in \R$, then:

    $$\lim_{n \to \infty} \int_{X} f_n \, d\mu = \int_{X} \lim_{n \to \infty} f_n \, d\mu$$

    (We are using the DCT with $g = M$)
\end{fremark}

\begin{fcorollary}[Dominated Convergence for series]
    Let $\{f_n\}_{n \in \N}$ be a sequence of measurable functions $f_n: X \to \R$,
    s.t $f_n \in \Lcur^1(X, \M, \mu)$. If $\sum_{n} \int_{X} |f_n| \, d\mu < \infty$,
    then:
    
    $$\int_{X} \sum_{n} f_n \, d\mu = \sum_{n} \int_{X} f_n \, d\mu$$
\end{fcorollary}

\section{Comparison between Riemann and Lebesgue integrals}

\begin{ftheorem}
    Let $I = [a, b] \subset \R$ be a closed interval, and $f: I \to \R$.
    If $f$ is \textbf{Riemann integrable} on $I$, then $f$ is \textbf{Lebesgue integrable} on $I$,
    i.e., $f \in \Lcur^1(I, \Lcur(I), \lambda)$, and the two integrals coincide:

    $$\int_{I} f \, d\lambda = \int_{a}^{b} f(x) \, dx$$

\end{ftheorem}

\vspace{1em}

\begin{ftheorem}
    Let $I = (\alpha, \beta)$, such that $-\infty \leq \alpha < \beta \leq \infty$.
    If $|f|$ is \textbf{Riemann integrable} on $I$ (in the generalized sense), 
    then $f$ is \textbf{Lebesgue integrable} on $I$:

    $$\int_{I} f \, d\lambda = \int_{\alpha}^{\beta} f(x) \, dx$$
\end{ftheorem}

\begin{fremark}
    If the generalized Riemann integral of $|f|$ diverges, then:

    $$\int_{I} |f| \, d\lambda = \infty$$

    but $\int_{I} f \, d\lambda$ is not defined (unless $f = \pm |f|$)
    and:

    $$\int_{\alpha}^{\beta} f(x) \, dx \text{ and } \int_{I} f \, d\lambda$$

    are not related.
\end{fremark}

\section{Spaces of integrable functions}

For a $(X, \M, \mu)$ complete measure space, we already know that
$\Lcur^1(X, \M, \mu)$ is a vector space. We can also define a distance
in this space:

$$d(f, g) = \int_{X} |f - g| \, d\mu$$

Immediately, we have that:

\begin{itemize}
    \item \textbf{Symmetry}: $d(f, g) = d(g, f)$
    
    \item \textbf{Triangle inequality}: $d(f, g) \leq d(f, h) + d(h, g)$
    
    \item \textbf{Non-negativity}: $d(f, g) \geq 0$
\end{itemize}

But notice that $d(f, g) = 0$ does not imply $f = g$ (only a.e.).
This means that $d(f, g)$ is a \textbf{pseudo-distance}.\\

To solve this, we can define an equivalence relation:

$$f \sim g \iff f = g \text{ a.e.}$$

With this equivalence relation, we can define the following space:

\begin{fdefinition}
    We define the space $L^1(X, \M, \mu)$ as:

    $$L^1(X, \M, \mu) = \{[f]: f \in \Lcur^1(X, \M, \mu)\}$$

    where $[f]$ is the equivalence class of $f$ defined as:

    $$[f] = \{g: g = f \text{ a.e.}\}$$
\end{fdefinition}

\begin{fremark}
    We can define the distance in $L^1(X, \M, \mu)$ as:

    $$d([f], [g]) = \int_{X} |f - g| \, d\mu$$

    This distance is well-defined, and it is a true distance.
    Then, $(L^1(X), d)$ is a metric space.
\end{fremark}

\begin{note}
We understand that elements of $L^1$ are functions: instead of $[u]$,
we work with a representant $u$, and we can \textbf{only} use 
operations/properties that are \textbf{independent of the representant}.
\end{note}

\begin{fexample}
    $X = (0, 1)$, we work on $(X, \Lcur(X), \lambda)$. If we take
    $u \in L^1(X)$, we have the following:

    \vspace{1em}
    \begin{itemize}
        \item $u \geq 0$ in $X$: \textbf{NOT} well-defined
        \vspace{1em}

        \item $u \geq 0$ a.e. on $X$: \textbf{GOOD}
        \vspace{1em}

        \item $u(1/2)$: \textbf{NOT} well-defined
        \vspace{1em}

        \item $\int_{[0, 1/2]} u \, d\lambda$: \textbf{GOOD}
    \end{itemize}
\end{fexample}

\vspace{1em}

\begin{fdefinition}
    Let $f: X \to \R$ be a measurable function. We say it is
    \textbf{essentially bounded} if:

    $$\exists M \in \R: |f(x)| \leq M \text{ a.e. on } X$$

    i.e.:

    $$\mu(\{x \in X: |f(x)| > M\}) = 0$$
\end{fdefinition}

\begin{example}
    Two examples:

    $$f(x) = \begin{cases}
        \infty & \text{if } x = 0\\
        0 & \text{if } x \in (0, 1] 
    \end{cases} \text{ is essentially bounded}
    $$

    $$g(x) = \begin{cases}
        0 & \text{if } x = 0\\
        1/x & \text{if } x \in (0, 1]
    \end{cases} \text{ is not essentially bounded}
    $$

\end{example}

\begin{fdefinition}
    If $f: X \to \R$ is essentially bounded, we define the
    \textbf{essential supremum} of $f$ as:

    $$\esssup f := \inf\{M \in \R: \mu(\{f > M\}) = 0\}$$

\end{fdefinition}

\vspace{1em}

\begin{fdefinition}
    We define the space $\Lcur^{\infty}(X, \M, \mu)$ as:

    $$\Lcur^{\infty}(X, \M, \mu) = \{f: X \to \R: f \text{ is essentially bounded}\}$$

    We can also define the space $L^{\infty}(X, \M, \mu)$ as:

    $$L^{\infty}(X, \M, \mu) = \{[f]: f \in \Lcur^{\infty}(X, \M, \mu)\}$$

    where $[f]$ is the equivalence class of $f$ defined as:

    $$[f] = \{g: g = f \text{ a.e.}\}$$
\end{fdefinition}

\begin{fremark}
    One can prove that $L^{\infty}(X, \M, \mu)$ is a vector space, with 
    the distance:

    $$d([f], [g]) = \esssup |f - g|$$
\end{fremark}
