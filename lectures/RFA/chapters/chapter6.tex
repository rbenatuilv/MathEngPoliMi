\chapter{Absolutely continuous functions and Functions of bounded variations}

\section{Fundamental theorems of calculus}

Let $(X, \Lcur(X), \lambda)$ be a complete measure space, such that 
$X = \R$ or $X = I \subset \R$ an interval.
Take $f \in L^1(a, b)$. We can define the \textbf{integral function}:

$$F(x) = \int_[a, x] f \, d\mu = \int_a^x f(t) \, dt$$

If $f \in C([a, b])$, then:

\begin{itemize}
    \item $F \in C^1([a, b])$
    \item $F'(x) = f(x)$
    \item $F(x) - F(y) = \int_y^x f(t) \, dt$
\end{itemize}

What if only $f \in L^1(a, b)$? 

\subsection{1st Fundamental Theorem of Calculus}

\begin{ftheorem}[1st Fundamental Theorem of Calculus]
    Let $f \in L^1(a, b)$. If we define:
    $$F(x) = \int_a^x f(t) \, dt$$
    then:
    \vspace{1em}
    \begin{itemize}
        \item $F$ is differentiable at a.e. $x \in [a, b]$
        \vspace{1em}
        \item $F'(x) = f(x)$ a.e. $x \in [a, b]$
    \end{itemize}
\end{ftheorem}

\begin{fexample}
    Take $[a, b] = [-1, 1]$ and:

    $$f(x) = \mathcal{H}(x) = \begin{cases}
        0 & x \leq 0 \\
        1 & x > 0
    \end{cases}$$

    This is the Heaviside function. Notice that $\mathcal{H} \in L^1(-1, 1)$.
    Now:

    $$F(x) = \int_{-1}^x \mathcal{H}(t) \, dt = \begin{cases}
        0 & x \leq 0 \\
        x & x > 0
    \end{cases}
    $$

    Also, if we define:

    $$f(x) = \begin{cases}
        \mathcal{H}(x) & x \notin \Q\\
        \infty & x \in \Q
    \end{cases}$$

    we get the same $F$.
\end{fexample}

\begin{note}
    For the proof, we need a deep result due to Lebesgue. We go back to
    $\Lcur^1([a, b])$.
\end{note}

\vspace{1em}

\begin{fdefinition}
    Le t $f \in \Lcur^1([a,b])$. we say $x \in [a, b]$ is a \textbf{Lebesgue point}
    for $f$ if:

    $$ \lim_{h \to 0} \frac{1}{h} \int_{x}^{x+h} |f(t) - f(x)| \, dt = 0$$

    Note that if $x = a$ then $h \to 0^+$ and if $x = b$, then $h \to 0^-$.
\end{fdefinition}

\begin{fremark}
    If $x$ is a LP, then:

    $$0 = \lim_{h \to 0} \frac{1}{h} \int_{x}^{x + h} |f(t) - f(x)| \, dt$$
    $$ \geq \lim_{h \to 0} \left| \frac{1}{h} \int_x^{x+h} (f(t) - f(x)) \, dt \right|$$
    $$ = \left| \left( \lim_{h \to 0} \int_x^{x+h} f(t) \, dt \right) - f(x) \right|$$

    i.e., LP is related with the validity of a local mean value theorem at $x$
\end{fremark}

\vspace{1em}

\begin{fremark}
    We have the following:
    \vspace{1em}
    \begin{itemize}
        \item $f$ is continuos $\implies$ $x$ is a LP.
        \vspace{1em}
        \item $f \in C([a, b]) \implies$ every $x \in [a, b]$ is a LP.
        \vspace{1em}
        \item Take $\mathcal{H}(x)$, then $x = 0$ is not a LP.
    \end{itemize}
\end{fremark}

\vspace{1em}

\begin{ftheorem}[Lebesgue]
    Let $f \in \Lcur^1([a, b])$. Then, a.e. $x \in [a, b]$ is a Lebesgue point.
\end{ftheorem}

\begin{fremark}
    By consequence of the theorem, it makes sense to consider Lebesgue points in 
    $L^1$. Indeed, changing the representative of the function class in
    $L^1$ maintains the same set of Lebesgue points up to a negligible set.
\end{fremark}

\vspace{1em}

\begin{note}
    To prove the \textbf{1st fund. thm.}, we will show that:
    \begin{itemize}
        \item $F$ is differentiable at $x$.
        \item $F'(x) = f(x)$
    \end{itemize}

    for all $x$ Lebesgue points for $f$.
\end{note}

\vspace{1em}

\begin{proof}[Proof: (1st fund. thm.)]
    Take $x \in [a, b]$ a LP of $f$. Then:

    $$ 0 \leq \left| \frac{F(x+h) - F(x)}{h} - f(x)\right| $$
    $$= \left| \frac{\int_a^{x+h} f(t) \, dt - \int_a^x f(t) \, dt}{h} - f(x) \right|$$
    $$= \left| \frac{1}{h} \int_x^{x+h} f(t) \, dt - \frac{1}{h} \int_x^{x+h} f(x) \, dt \right|$$
    $$\leq \frac{1}{h} \int_x^{x+h} |f(t) - f(x)| \, dt \to 0 \quad \text{as } h \to 0$$

    because $x$ is a LP.

\end{proof}

\begin{fremark}
    Let us try to reverse the point of view: take $g: [a, b] \to \R$, and assume that
    $g$ is differentiable a.e. in $[a, b]$, and that $g' \in L^1([a, b])$. Is 
    $g$ related with $\int_a^x g'(t) \, dt$?. The answer is \textbf{NO!}
\end{fremark}

\begin{fexample}
    $\mathcal{H}: [-1, 1] \to \R$ and notice that:

    $$\mathcal{H}'(x) 0 \begin{cases}
        \nexists & x = 0\\
        0 & x \neq
    \end{cases}$$

    We have that $\mathcal{H}' = 0$ a.e. in $[-1, 1]$, and $0 \in L^1([-1, 1])$.
    But:

    $$\mathcal{H}(1) - \mathcal{H}(0) = 1 - 0 = 1 \neq 0 = \int_{-1}^1 0 \, dt = \int_{-1}^1 \mathcal{H}'(t) \, dt$$

    Other example with the Cantor-Vitali function:

    $$g(x) = v(x), \quad \text{s.t. } v(0) = 0, v(1) = 1 \quad \text{and constant outside the Cantor set}$$

    Then, $v$ is differentiable and $v'(x) = 0$ a.e., but we can notice that the same thing
    as before happens.
\end{fexample}

\vspace{1em}

\begin{fdefinition}
    Let $I$ be an interval. We say that $f: I \to \R$ is an 
    \textbf{absolutely continuous function}, $f \in AC(I)$, if:\\

    $\forall \varepsilon > 0, \exists \delta$ s.t., $\forall n \in \N$, 
    $\forall$ family of $n$ disjoint subintervals of $I$, i.e., $(a_i, b_i) \subset I$ s.t.
    $... b_{i - 1} \leq a_i < b_i \leq a_{i+1} < ...$ we have that:

    $$\lambda \left( \bigcup_{i=1}^n (a_i, b_i) \right) < \delta \implies \sum_{i=1}^n |f(b_i) - f(a_i)| \leq \varepsilon$$  

\end{fdefinition}

\begin{fremark}
    Recall that $f$ is uniformly continuous (UC) if:

    $$\forall \varepsilon > 0, \exists \delta > 0 \text{ s.t. } \forall x, y \in I$$
    $$|x - y| < \delta \implies |f(x) - f(y)| < \varepsilon$$

    (The choice of $\delta$ is independent of $x, y$)\\

    Then:

    $$UC(I) \supset AC(I)$$
    
    Recall that $f$ is Lipschitz continuous if $\exists L > 0$ s.t.:

    $$\forall x, y \in I, |f(x) - f(y)| \leq L|x - y|$$

    Then:

    $$Lip(I) \subset AC(I)$$

    We will see that:

    $$Lip(I) \subsetneq AC(I) \subsetneq UC(I)$$

    We will also see that, as $g' \in C \iff g \in C^1$, we have that:

    $$g' \in L^1 \iff g \in AC$$
\end{fremark}

\subsection{2nd Fundamental Theorem of Calculus}

\begin{ftheorem}[2nd Fundamental Theorem of Calculus]
    Let $g: [a, b] \to \R$. The following are equivalent:
    \vspace{1em}
    \begin{enumerate}[label=(\roman*)]
        \item $g \in AC([a, b])$
        \vspace{1em}
        \item $g$ is differentiable a.e. in $[a, b]$, $g' \in L^1([a, b])$
        and:

        $$g(x) - g(y) = \int_y^x g'(t) \, dt \quad \forall x, y \in [a, b]$$
    \end{enumerate}
\end{ftheorem}

\begin{fcorollary}
    $f \in L^1([a, b]) \implies F(x) = \int_a^x f(t) \, dt \in AC([a, b])$
\end{fcorollary}

\begin{note}
    To prove one implication of the theorem, we will need some few extra 
    results.
\end{note}

\vspace{1em}

\begin{ftheorem}[Absolute continuity of the integral function]
    Let $f \in L^1([a, b])$. Then, $\forall \varepsilon > 0$, $\exists \delta > 0$ s.t.:

    $$\begin{cases}
        E \in \M\\
        \mu(E) < \delta
    \end{cases} \implies \int_{E} |f| \, d\mu < \varepsilon$$
    
\end{ftheorem}

\begin{proof}
    By contradiction: assume that $\exists \varepsilon > 0$ s.t. $\forall \delta > 0$,
    $\exists E \in \M$ s.t. $\mu(E) < \delta$ and $\int_E |f| \, d\mu \geq \varepsilon$.\\

    In particular, $\delta = 1/2^n \to 0$, $E_n = E_{\delta_n}$ and:

    $$F_n = \bigcup_{k=n}^{\infty} E_n = E_n \cup F_{n+1}, \quad F = \lim_{n \to \infty} F_n$$

    Then:
    \begin{enumerate}
        \item $$(F_{n+1} \subset F_n) \implies \{F_n\} \downarrow F$$
        \item $$\forall n, \quad  \mu(F_n) \leq \sum_{k=n}^{\infty} \mu(E_k) \leq \sum_{k=n}^{\infty} \delta_k = \sum_{k=n} \frac{1}{2^k} = 2^{-n+1}$$
        \item $$\nu(F_n) = \int_{F_n} |f| \, d\mu \geq \int_{E_n} |f| \, d\mu \geq \epsilon \quad \forall n$$
        Moreover:
        $$\nu(F_1) = \int_{F_1} |f| \, d\mu \leq \int_X |f| \, d\mu < \infty$$
    \end{enumerate}

    Use continuity of measures:
    $$(1) + (2) \implies \nu(F) = \lim_{n \to \infty} \nu(F_n) = 0$$
    $$(1) + (3) \implies \nu(F) = \lim_{n \to \infty} \nu(F_n) \geq \varepsilon > 0$$

    Contradiction, since $\nu(F) = 0$.
\end{proof}

\begin{fremark}
    As a consequence, we have:

    $$f \in L^1([a, b]) \implies F(x) = \int_a^x f(t) \, dt \in AC([a, b])$$
\end{fremark}

\begin{proof}
    Take $\varepsilon > 0$, and $\delta = \delta(\varepsilon)$ as in the theorem.
    I know:

    $$\begin{cases}
        \forall E \in \Lcur([a, b])\\
        \lambda(E) < \delta
    \end{cases} \implies \int_E |f| \, d\lambda < \varepsilon$$

    Take $E = \bigcup_{i=1}^n (a_i, b_i)$, s.t $(a_i, b_i)$ disjoint intervals.
    Then:

    $$\sum_{i=1}^n |F(b_i) - F(a_i)| = \sum_{i=1}^n \left| \int_{a_i}^{b_i} f(t) \, dt \right| \leq \sum_{i=1}^n \int_{a_i}^{b_i} |f| \, dt$$
    $$= \int_{\bigcup_{i=1}^n (a_i, b_i)} |f| \, dt < \varepsilon$$

\end{proof}

\begin{fexample}[(AC $\nRightarrow$ Lip)]
    Consider $g(x) = \sqrt{x}$ in $[0, 1]$. Then:

    $$\sqrt{x} = \int_0^x \frac{1}{2\sqrt{t}} \, dt$$

    and $g \in AC([0, 1])$. But notice that $g \notin Lip([0, 1])$.

    $$\left| \frac{\sqrt{x} - \sqrt{0}}{x - 0} \right| \nleq C$$

    for any $C > 0$, as $x \to 0$.
\end{fexample}

\begin{fexample}[(UC $\nRightarrow$ AC)]
    Consider:
    $$f(x) = \begin{cases}
        x \sin(1/x) & x \neq 0\\
        0 & x = 0
    \end{cases}$$

    It is continuous in $[0, 1] \implies f \in UC([0, 1])$. But notice that
    $f \notin AC([0, 1])$. Indeed:

    $$f'(x) = \sin(1/x) - \frac{1}{x} \cos(1/x)$$

    and $1/x \cos(1/x)$ is not integrable in $[0, 1]$, i.e., $f' \notin L^1([0, 1])$.
\end{fexample}

\section{AC functions and weak derivatives}

\begin{fproposition}[Integration by parts in AC]
    Let $u: [a, b] \to \R$. Then, $u \in AC([a, b])$ if and only if:
    \vspace{1em}
    \begin{itemize}
        \item $u \in C([a, b])$
        \vspace{1em}
        \item $u$ is differentiable a.e. in $[a, b]$
        \vspace{1em}
        \item $u' \in L^1([a, b])$
        \vspace{1em}
        \item $$\int_a^b u' \varphi dx = - \int_a^b u \varphi' dx \quad \forall \varphi \in C_0^{\infty}([a, b])$$
    \end{itemize}
\end{fproposition}

\vspace{1em}

\begin{fdefinition}[Weak derivative]
    Let $u \in L^1(a, b)$. We say that $u \in W^{1,1}(a, b) \iff \exists w \in L^1(a, b)$ s.t.:

    $$\int_a^b u \varphi' dx = - \int_a^b w \varphi dx \quad \forall \varphi \in C_0^{\infty}(a, b)$$

    Such $w$ is called the \textbf{weak derivative} of $u$.
\end{fdefinition}

\begin{fremark}
    Both $u$ and $w = u'$ are equivalence classes of functions, i.e., $u \sim v \iff u = v$ a.e.
Properties should be independent of the representative.
\end{fremark}

\begin{fremark}
    If such a $w$ exists, it is unique (up to a.e. equivalence). Indeed, 
    assume that $w_1, w_2$ are weak derivatives of $u$. Then:
    
    $$\int_a^b (w_1 - w_2) \varphi dx = 0 \quad \forall \varphi \in C_0^{\infty}(a, b)$$
    $$\implies w_1 - w_2 = 0 \text{ a.e.}$$
\end{fremark}

\begin{fremark}
    In principle, the pointwise and weak derivatives are different objects,
    and the notation $u'$ may be misleading. But we know that if $u \in AC([a, b])$
    they coincide.    
\end{fremark}

\begin{fremark}
    In principle, the definition of weak derivatives can be 
    extended (measures, distributions). Take:

    $$\mathcal{H}(x) = \begin{cases}
        0 & x \leq 0\\
        1 & x > 0
    \end{cases}$$

    Then:

    $$- \int_{-1}^1 \mathcal{H}(x) \varphi'(x) \, dx = - \int_0^1 \varphi'(x) \, dx = \varphi(0) - \varphi(1)$$
    $$= \varphi(0) = \int_{[-1, 1]} \varphi(x) \,  d\delta_0$$

    where $\delta_0$ is the Dirac delta function. This suggest that:

    $$\mathcal{H}' = \delta_0 \text{ weakly}$$
    $$\mathcal{H}' = 0 \text{ pointwise}$$
\end{fremark}

\vspace{1em}

\begin{ftheorem}
    $u \in AC([a, b]) \iff u \in W^{1,1}(a, b)$
\end{ftheorem}

\begin{proof}
    The proof goes as follows:
    \begin{itemize}
        \item[($\Rightarrow$)] Already proved.
        \item[($\Leftarrow$)] Assume that $u'$ weak derivative of $u$, $u' \in L^1(a, b)$.
        Then:
        $$z(x) = \int_a^x u'(t) \, dt, \quad z \in AC$$
        
        We can show that $u = z + c$ for some constant $c$.
    \end{itemize}
\end{proof}

