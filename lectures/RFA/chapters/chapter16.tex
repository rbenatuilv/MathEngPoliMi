\chapter{Spectral theory}

\begin{note}
    We will consider $E$ Banach, $T \in \Lcur(E, E) = \Lcur(E)$, and the problem:
    $$Tx = \lambda x \iff (T - \lambda I)x = 0$$
\end{note}

\begin{fdefinition}
    We define the following concepts:
    \vspace{1em}
    \begin{itemize}
        \item The \textbf{resolvent set} of $T$ is:
        $$\rho(T) = \{ \lambda \in \R: T - \lambda I: E \to E \text{ is bijective} \}$$

        \vspace{1em}
        \item The \textbf{spectrum} of $T$ is:
        $$\sigma(T) = \R \setminus \rho(T)$$

        \vspace{1em}
        \item $\lambda$ is an \textbf{eigenvalue} of $T$ if:
        $$Ker(T - \lambda I) \neq \{0\}$$

        where $Ker (T - \lambda I)$ is called the \textbf{eigenspace} corresponding to $\lambda$.
        Also:
        $$EV(T) = \{\text{eigenvalues of } T\} \subset \R$$
    \end{itemize}
\end{fdefinition}

\begin{fremark}
    Note that:
    $$EV(T) \subset \sigma(T)$$

    as $\lambda \in EV(T) \iff T - \lambda I$ is not injective. Also, note that if $dim E < \infty$, 
    then $EV(T) = \sigma(T)$. If $E$ has infinite dimension, then the inclusion may be strict.
\end{fremark}

