\chapter{Lebesgue spaces $L^p(X)$}

\section{Definition of $L^p(X)$}

\begin{fdefinition}
    Let $(X, \M, \mu)$ be a complete measure space, and $p \in [1, \infty]$.
    We define the following:
    \vspace{1em}
    \begin{enumerate}
        \item $\Lcur^p(X, \M, \mu) := \left\{ f: X \to \R \mid f \text{ is } \M\text{-measurable and } \int_X |f|^p \, d\mu < \infty \right\}$.
        \vspace{1em}
        \item $u, v \in \Lcur^p(X, \M, \mu)$, $u \sim v$ $\iff$ $u = v$ a.e.
        \vspace{1em}
        \item $[f]_p := \left\{ g \in \Lcur^p(X, \M, \mu) \mid f \sim g \right\}$.
    \end{enumerate}
    \vspace{1em}
    Finally, we define the $L^p$-space as follows:

    $$L^p(X, \M, \mu) := \Lcur^p(X, \M, \mu) / \sim \, = \left\{ [f]_p \mid f \in \Lcur^p(X, \M, \mu) \right\}$$

    where $\sim$ is the equivalence relation defined above. We also define
    the norm as follows:

    $$\norm{f}_{L^p} = \norm{f}_p = \begin{cases}
        \left( \int_X |f|^p \, d\mu \right)^{1/p} & \text{if } 1 \leq p < \infty \\
        \esssup_{x \in X} |f(x)| & \text{if } p = \infty
    \end{cases}$$

    and $d_p(f, g) = \norm{f - g}_p$.
\end{fdefinition}

\begin{fexample}
    Notice that if $(X, \M, \mu) = (\N, \power(\N), \mu_{\#})$, then:
    $$L^p(\N, \power(\N), \#) = \ell^p$$

    For $1 \leq p < \infty$, we have:

    $$\ell^p = \left\{ (a_n)_{n \in \N} \mid \sum_{n \in \N} |a_n|^p < \infty \right\}$$

    with norm:

    $$\norm{(a_n)}_p = \left( \sum_{n \in \N} |a_n|^p \right)^{1/p}$$

    For $p = \infty$, we have:

    $$\ell^{\infty} = \left\{ (a_n)_{n \in \N} \mid \sup_{n \in \N} |a_n| < \infty \right\}$$

    with norm:

    $$\norm{(a_n)}_{\infty} = \sup_{n \in \N} |a_n|$$
\end{fexample}

\begin{note}
    Our plan is to show that $L^p(X, \M, \mu)$ is a Banach space, i.e.:
    \begin{enumerate}
        \item $L^p(X, \M, \mu)$ is a vector space.
        \item $\norm{\cdot}_p$ is a norm.
        \item $L^p(X, \M, \mu)$ is complete.
    \end{enumerate}
\end{note}

\section{$L^p$-spaces are vector spaces}

\begin{flemma}
    Let $p \in [1, \infty)$, $a, b \in \R, a, b \leq 0$. Then:

    $$(a + b)^p \leq 2^{p - 1} (a^p + b^p)$$
\end{flemma}

\begin{proof}[Proof (exercise)]
    For $a \neq 0$, $t = b / a$, we have to show that:

    $$\frac{(1 + t)^p}{1 + t^p} \leq 2^{p - 1} \quad \forall t \leq 0$$
    
\end{proof}

\begin{ftheorem}
    Let $p \in [1, \infty)$, then $L^p(X)$ is a vector space
\end{ftheorem}

\begin{proof}
    Given $u, v \in L^p(X), \alpha \in \R$, we have to show that:

    \begin{enumerate}
        \item $u + v \in L^p(X)$
        \item $\alpha u \in L^p(X)$
    \end{enumerate}

    \begin{enumerate}
        \item We have:

        $$\int_X |u + v|^p \, d\mu \leq \int_X (|u| + |v|)^p \, d\mu \leq 2^{p - 1} \left( \int_X |u|^p \, d\mu + \int_X |v|^p \, d\mu \right) < \infty$$

        \item We have:

        $$\int_X |\alpha u|^p \, d\mu = \int_X |\alpha|^p |u|^p \, d\mu = |\alpha|^p \int_X |u|^p \, d\mu < \infty$$

    \end{enumerate}
\end{proof}

\section{$(L^p(X), \norm{\cdot}_p)$ are normed spaces}

\begin{fdefinition}[Conjugated exponent]
    For every $1 \leq p \leq \infty$, the \textbf{conjugated exponent} of $p$,
    denoted by $q \in [1, \infty]$, satisfies:

    $$\frac{1}{p} + \frac{1}{q} = 1$$
    
\end{fdefinition}

\begin{flemma}[Young's inequality]
    Let $p, q \in (1, \infty)$ be conjugated exponents. Then, for every $a, b \geq 0$:

    $$ab \leq \frac{a^p}{p} + \frac{b^q}{q}$$
\end{flemma}

\begin{proof}
    Notice that $\ln(x)$ is a concave function. Then:

    $$\ln(\frac{a^p}{p} + \frac{b^q}{q}) \geq \frac{1}{p} \ln(a^p) + \frac{1}{q} \ln(b^q)$$
    $$= \ln((a^p)^{1/p}) + \ln((b^q)^{1/q}) = \ln(a) + \ln(b) = \ln(ab)$$

\end{proof}

\begin{note}
    As a consequence of Young's inequality, we have the
    following inequality:
\end{note}

\begin{flemma}[Hölder's inequality]
    Let $p, q \in [1, \infty]$ be conjugated exponents, $(X, \M, \mu)$ be a complete measure space,
    and $u, v$ measurable functions. Then:

    $$\norm{uv}_1 \leq \norm{u}_p \norm{v}_q$$
\end{flemma}

\begin{proof}
    We will prove it for $p, q \in (1, \infty)$. For $p = 1, q = \infty$, it
    is left as an exercise.\\
    
    We separate in cases:
    \begin{itemize}
        \item If $\norm{u}_p = 0$, then $u = 0$ a.e., and $uv = 0$ a.e., meaning that
        $$\norm{uv}_1 = 0$$

        (The same applies if $\norm{v}_q = 0$)

        \item If $\norm{u}_p \cdot \norm{v}_q = \infty$, then the inequality is trivial.
        
        \item For $0 < \norm{u}_p, \norm{v}_q < \infty$, we apply the Young inequality for:
        
        $$a = \frac{|u(x)|}{\norm{u}_p}, \quad b = \frac{|v(x)|}{\norm{v}_q}$$

        We have:

        $$\frac{|u(x)| \cdot |v(x)|}{\norm{u}_p \norm{v}_q} = a b \leq \frac{1}{p} \frac{|u(x)|^p}{\norm{u}_p^p} + \frac{1}{q} \frac{|v(x)|^q}{\norm{v}_q^q}$$

        We integrate to get:

        $$\frac{\norm{uv}_1}{\norm{u}_p \norm{v}_q} \leq \frac{1}{p} \frac{\norm{u}_p^p}{\norm{u}_p^p} + \frac{1}{q} \frac{\norm{v}_q^q}{\norm{v}_q^q} = 1$$

        $$\implies \norm{uv}_1 \leq \norm{u}_p \norm{v}_q$$
    \end{itemize}
\end{proof}

\subsection{Inclusion of $L^p$ spaces}

\begin{ftheorem}
    Let $\mu(X) < \infty$, $1 \leq p \leq q \leq \infty$. Then:

    $$L^q(X) \subset L^p(X)$$

    More precisely, $\exists C > 0$ s.t.:

    $$\norm{u}_p \leq C \norm{u}_q$$
\end{ftheorem}

\begin{ftheorem}[Interpolation]
    Let $1 \leq p < q \leq \infty$. Then:

    $$L^r(X) \subset L^p(X) \cap L^q(X), \quad \forall p \leq r \leq q$$
\end{ftheorem}

\subsection{Minkowski's inequality}

\begin{ftheorem}[Minkowski's inequality]
    Let $p \in [1, \infty]$, $(X, \M, \mu)$ be a complete measure space, and $u, v \in L^p(X)$. Then:

    $$\norm{u + v}_p \leq \norm{u}_p + \norm{v}_p$$
\end{ftheorem}

\begin{proof}
    We will prove it for $p \in (1, \infty)$. For $p = 1, p = \infty$, it is left as an exercise.\\

    We have:

    $$\norm{u + v}_p^p = \int_X |u + v|^p \, d\mu = \int_X |u + v| |u + v|^{p - 1} \, d\mu$$
    $$ \leq \int_X |u| |u + v|^{p - 1} \, d\mu + \int_X |v| |u + v|^{p - 1} \, d\mu$$

    For the first term, we have:

    $$\int_X |u| |u + v|^{p - 1} \, d\mu \leq \norm{u}_p \left( \int_X |u + v|^{(p - 1)q} \, d\mu \right)^{1/q}$$
    $$ \leq \norm{u}_p \norm{u + v}_p^{p/q} = \norm{u}_p \norm{u + v}_p^{p - 1}$$

    Analogously, for the second term, we have:

    $$\int_X |v| |u + v|^{p - 1} \, d\mu \leq \norm{v}_p \norm{u + v}_p^{p - 1}$$

    and finally, we substitute back to get:

    $$\norm{u + v}_p^p \leq \norm{u}_p \norm{u + v}_p^{p - 1} + \norm{v}_p \norm{u + v}_p^{p - 1}$$

    and we divide by $\norm{u + v}_p^{p - 1}$ to get:

    $$\norm{u + v}_p \leq \norm{u}_p + \norm{v}_p$$
\end{proof}

\begin{fcorollary}
    ($L^p(X), \norm{\cdot}_p$) is a normed space for $p \in [1, \infty]$
\end{fcorollary}

\section{Completeness of $L^p$-spaces}

\begin{ftheorem}[Riesz-Fischer]
    Let $p \in [1, \infty]$, $(X, \M, \mu)$ be a complete measure space. Then:

    $$L^p(X) \text{ is a Banach space}$$
\end{ftheorem}

\begin{proof}
    The only thing left to show is that $L^p(X)$ is complete. We will
    use the characterization of Banach spaces in terms of absolutely
    convergent series.\\

    Let us suppose that $\{f_n\}_n \subset L^p(X)$ is an absolutely
    convergent series, i.e.:

    $$\sum_{n = 1}^{\infty} \norm{f_n}_p = M < \infty$$

    Introduce $g_k(x) = \sum_{n = 1}^k |f_n(x)|$. We have that, for every
    $x \in X$, $\{g_k(x)\}_{k \in \N}$ is a non-decreasing sequence. Then:

    $$g(x) = \lim_{k \to \infty} g_k(x) = \sum_{n = 1}^{\infty} |f_n(x)|$$

    is well-defined for every $x \in X$. We have to show that $g \in L^p(X)$.\\

    Notice that:

    $$\norm{g_k}_p = \norm{\sum_{n = 1}^{k} |f_n|}_p \leq \sum_{n = 1}^{k} \norm{f_n}_p \leq$$
    $$ \leq \sum_{n = 1}^{\infty} \norm{f_n}_p = M$$

    where $M$ is a constant (since the series is absolutely convergent). Then, $g_k \in L^p(X)$
    for every $k \in \N$.\\

    Then, by the monotone convergence theorem, we have:

    $$\int_X g^p \, d\mu = \int_X \left( \lim_{k \to \infty} g_k \right)^p \, d\mu = \lim_{k \to \infty} \int_X g_k^p \, d\mu$$
    $$= \lim_{k \to \infty} \norm{g_k}_p^p \leq M^p < \infty$$

    Then, $g \in L^p(X)$, meaning that $g(x) \leq \infty$ a.e., which implies
    that:

    $$\sum_{n = 1}^{\infty} |f_n(x)| < \infty \quad \text{a.e.}$$

    Since $X$ is complete, we have that $\sum_{n = 1}^{\infty} f_n(x)$ converges a.e.
    Then:

    $$s(x) = \sum_{n = 1}^{\infty} f_n(x)$$

    is well-defined for every $x \in X$. And we proved that $s_k(x) \to s(x)$ for a.e $x \in X$.\\

    To conclude, we apply the dominated convergence theorem:

    \begin{itemize}
        \item $|s_k(x) - s(x)| \to 0$ a.e.
        \item $$ |s_k - s|^p = \left| \sum_{n = k + 1}^{\infty} f_n \right|^p \leq \left( \sum_{n=k+1}^{\infty} |f_n| \right)^p$$
        $$ \leq (g)^p \in L^1$$
    \end{itemize}

    These conditions imply that:

    $$\int_X |s_k - s|^p \, d \mu \to 0$$

    that is, convergence in $L^p$.
\end{proof}

\begin{fexample}
    We know that the following are Banach spaces:
    \vspace{1em}
    \begin{enumerate}
        \item $(\R^N, \text{any norm})$
        \vspace{1em}
        \item $(C([a, b]), \norm{\cdot}_{\infty})$
        \vspace{1em}
        \item $(L^p(X), \norm{\cdot}_p)$
        \vspace{1em}
        \item $(L^{\infty}, \norm{\cdot}_{\infty})$
    \end{enumerate}
\end{fexample}

\vspace{1em}

\begin{fexample}
    Let $X = C([-1, 1])$, $\norm{u}_1 = \int_{-1}^{1} |u(x)| \, dx$. Then,
    let $u_n$:

    $$u_n(x) = \begin{cases}
        0 & \text{if } x \in [-1, 0) \\
        nx & \text{if } x \in [0, 1/n] \\
        1 & \text{if } x \in [1/n, 1]
    \end{cases}$$

    We have that $\{u_n\}_n \subset X$ is a Cauchy sequence with respect to
    the norm $\norm{\cdot}_1$. On the other hand:

    $$\norm{u_n - u_m}_{\infty} = \max_{-1 \leq x \leq 1} |u_n(x) - u_m(x)| = 1 - \frac{n}{m} \notto 0$$

    Moreover, we have that $\{u_n\}_n \subset L^1([-1, 1])$, s.t.
    $u_n \to \mathcal{H}$, which is not in $C([-1, 1])$.\\

    \textbf{Consequences:}
    \vspace{1em}
    \begin{enumerate}
        \item $\norm{\cdot}_1$ is not equivalent to $\norm{\cdot}_{\infty}$
        in $C([-1, 1])$.
        \vspace{1em}
        \item $(C([-1, 1]), \norm{\cdot}_1)$ is not a Banach space.
        \vspace{1em}
        \item $C([-1, 1])$ is a vector subspace of $L^1([-1, 1])$,
        but it is not closed, since the sequence $\{u_n\}_n \subset C([-1, 1])$ converges
        to a function that is not in $C([-1, 1])$.
    \end{enumerate}
\end{fexample}






 


