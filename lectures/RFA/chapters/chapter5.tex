\chapter{Types of convergence}

We have various types of convergence for sequences of measurable
functions:

\begin{fdefinition}
    Let $f_n: X \to \Rbar$ be a sequence of measurable functions,
    that converges to a function $f: X \to \Rbar$. We say that the
    convergence is a:
    \vspace{1em}
    \begin{itemize}
        \item \textbf{Pointwise convergence}:
        $$f_n(x) \to f(x) \quad \forall x \in X$$
        \item \textbf{Uniform convergence}:
        $$\sup_{x \in X} |f_n(x) - f(x)| \to 0$$
        \item \textbf{Convergence a.e.}:
        $$f_n(x) \to f(x) \quad \text{a.e. } x \in X$$
        \item \textbf{$L^1$-convergence}:
        $$\int_X |f_n - f| \, d\mu \to 0$$
        \item \textbf{$L^\infty$-convergence}:
        $$\esssup_X |f_n - f| \to 0$$
        \item \textbf{Convergence in measure}:
        $$\mu(\{x \in X: |f_n(x) - f(x)| \geq \epsilon\}) \to 0 \quad \forall \epsilon > 0$$
    \end{itemize}
\end{fdefinition}

\begin{fremark}
    Basic relations:
    $$\text{Uniform convergence} \Rightarrow \text{Pointwise convergence} \Rightarrow \text{Convergence a.e.}$$
    $$\text{Uniform convergence} \Rightarrow \text{$L^\infty$-convergence}$$
\end{fremark}

\begin{fexercise}
    Let $([0, 1], \Lcur([0, 1]), \lambda)$ be the Lebesgue measure space.
    Let:
    $$f_n(x) = e^{-nx} \quad 0 \leq x \leq 1$$
    $$g(x) = \begin{cases}
        1 & x = 0 \\
        0 & x \in (0, 1]
    \end{cases}$$

    $$f(x) = 0 \quad 0 \leq x \leq 1$$

    Show that:

    \begin{itemize}
        \item $f_n \to f$ a.e.
        \item $f_n \notto f$ pointwise
        \item $f_n \to g$ pointwise
        \item $f_n \notto g$ uniformly 
    \end{itemize}
\end{fexercise}
\vspace{1em}

\section{a.e. convergence and convergence in measure}

\begin{ftheorem}
    Let $\mu(X) < \infty$, $f_n, f$ measurable functions, a.e. finite in $X$.
    If $f_n \to f$ a.e., then $f_n \to f$ in measure.
\end{ftheorem}

\begin{fremark}
    if $\mu(X) = \infty$, then the theorem may not hold. For instance,
    consider $X = \R$, with the Lebesgue measure, and:
    $$f_n(x) = \chi_{[n, \infty)}(x) = \begin{cases}
        1 & x \geq n \\
        0 & x < n
    \end{cases}$$

    We can show that $f_n(x) \to 0$ a.e., but 
    $\lambda(\{f_n \geq 1/2\}) = \infty \; \forall n$ and thus
    $f_n \notto 0$ in measure.\\

    Also notice that convergence in measure \textbf{does not imply} 
    convergence a.e., even if $\mu(X) < \infty$. For instance, consider
    the \textbf{\say{typewriter sequence}}.
\end{fremark}

\vspace{1em}

\begin{ftheorem}
    Let $f_n, f$ be measurable functions, a.e. finite in $X$. If $f_n \to f$ in measure,
    then there exists a subsequence $f_{n_k}$ that converges to $f$ a.e.
\end{ftheorem}

\section{Convergence in $L^1$ and convergence in measure}

\begin{ftheorem}
    Let $f_n, f$ be measurable functions in $L^1(X, \M, \mu)$.
    If $f_n \to f$ in $L^1$, then $f_n \to f$ in measure.
\end{ftheorem}

\begin{proof}
    Assume by contradiction that $f_n \notto f$ in measure.
    Then $\exists \alpha > 0$ s.t.:

    $$\mu(\{x \in X: |f_n(x) - f(x)| \geq \alpha\}) \notto 0$$

    I.e., $\exists \epsilon > 0$ and a subsequence $f_{n_k}$ s.t.:

    $$\mu(\{x \in X: |f_{n_k}(x) - f(x)| \geq \alpha\}) \geq \epsilon \quad \forall k$$

    Let us call $E_k = \{x \in X: |f_{n_k}(x) - f(x)| \geq \alpha\}$.
    On the other hand, by assumption, $f_{n_k} \to f$ in $L^1$.
    But notice that:

    $$\int_X |f_{n_k} - f| \, d\mu \geq \int_{E_k} |f_{n_k} - f| \, d\mu \geq \alpha \mu(E_k) \geq \alpha \epsilon > 0$$

    Since $f_{n_k} \to f$ in $L^1$, we have that $\int_X |f_{n_k} - f| \, d\mu \to 0$. 
    But we have just shown that $\int_X |f_{n_k} - f| \, d\mu \geq \alpha \epsilon > 0$.
    This is a contradiction, and thus $f_n \to f$ in measure.
\end{proof}

\begin{fremark}
    In general, convergence in measure does not imply convergence in $L^1$.
    For instance, consider $X = [0, 1]$, $\M = \Lcur([0, 1])$, $\mu$ the Lebesgue measure,
    and $f_n(x) = n \chi_{[0, 1/n]}(x)$. We can show that $f_n \to 0$ in measure, but
    $\int_X |f_n - 0| \, d\mu = 1 \; \forall n$.
\end{fremark}

\section{Convergence in $L^1$ and a.e. convergence}

In general, they are not related. But we have 2 main results:
\textbf{Dominating convergence theorem} that we already
saw, and the \textbf{\say{Reverse Dominating Convergence Theorem}}, 
that states:

\begin{ftheorem}
    Let $f_n \to f$ in $L^1(X, \M, \mu)$, then there exists a subsequence
    $f_{n_k}$ that converges to $f$ a.e., and there exists a function $g \in L^1(X, \M, \mu)$
    s.t. $|f_{n_k}| \leq g$ a.e. $\forall k$.
\end{ftheorem}



