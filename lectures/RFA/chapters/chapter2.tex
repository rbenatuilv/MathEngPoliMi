\chapter{Measure Spaces}

\section{Measurable spaces}

\begin{fdefinition}
    Let $X$ be a non-empty set. A family $\mathcal{M} \subseteq \power(X)$ is a
    \textbf{$\sigma$-algebra} if:

    \vspace{1em}

    \begin{enumerate}
        \item[(i)] $\emptyset \in \mathcal{M}$.
        \vspace{1em}
        \item[(ii)] $E \in \mathcal{M} \implies E^c \in \mathcal{M}$.
        \vspace{1em}
        \item[(iii)] $\{E_n\}_{n \in \N} \subseteq \mathcal{M} \implies \bigcup_{n \in \N} E_n \in \mathcal{M}$.
    \end{enumerate}

    \vspace{1em}

    If instead of (iii) we have that $E_1, E_2 \in \M \implies \E_1 \cup E_2 \in \M$, then
    $\M$ is called an \textbf{algebra}.

\end{fdefinition}

\begin{fremark}
    If $\mathcal{M}$ is a $\sigma$-algebra, then we say that $(X, \mathcal{M})$ is a
    \textbf{measurable space}. Any set $E \in \mathcal{M}$ is called a
    \textbf{measurable set}.
\end{fremark}

\begin{example}
    Let $X \neq \emptyset$. Then:

    \begin{itemize}
        \item $\power(X)$ is a $\sigma$-algebra.
        \item $\{\emptyset, X\}$ is a $\sigma$-algebra.
        \item $\{\emptyset, E, E^c, X\}$ is a $\sigma$-algebra for any $E \subseteq X$.
        \item $X = \R$, $\mathcal{M} = \{E \subseteq \R: \; E \text{ is open}\}$ is NOT a $\sigma$-algebra.
    \end{itemize}
\end{example}

\begin{fproperties}
    Let $(X, \mathcal{M})$ be a measurable space. Then:

    \vspace{1em}

    \begin{enumerate}[label=(\roman*)]
        \item $X = \emptyset^c \in \M$
        
        \vspace{0.5em}
    
        \item $\M$ is also an algebra. Indeed, if $\{E_1, E_2\} \subseteq \M,\; E_n = \emptyset \; \forall n \geq 3$,
        then $E_1 \cup E_2 = \bigcup_{n \in \N} E_n \in \M$.

        \vspace{0.5em}

        \item $\{E_n\}_{n \in \N} \subseteq \M \implies \bigcap_{n} E_n \in \M$.
        
        \vspace{0.5em}

        \item $E, F \in \M \implies E\setminus F \in \M$
        
        \vspace{0.5em}

        \item $\Omega \subseteq X$. Then, the \textbf{restriction} of $\M$ to $\Omega$ is:
        
        $$\M |_{\Omega} := \{F \subseteq \Omega: F = E \cap \Omega, E \in \M\}$$

        Then, $(\Omega, \M |_{\Omega})$ is a measurable space.
    \end{enumerate}
\end{fproperties}

\section{Generation of a \texorpdfstring{$\sigma$}{sigma}-algebra}

\begin{ftheorem}
    Take any family $\mathcal{A} \subseteq \power(X)$. Then, it is well-defined
    the $\sigma$-algebra generated by $\mathcal{A}$, denoted by $\sigma_0(\mathcal{A})$,
    as the smallest $\sigma$-algebra containing $\mathcal{A}$. It is characterized by:

    \vspace{1em}

    \begin{enumerate}[label=(\roman*)]
        \item $\sigma_0(\mathcal{A})$ is a $\sigma$-algebra.
        
        \vspace{1em}

        \item $\mathcal{A} \subseteq \sigma_0(\mathcal{A})$.
        
        \vspace{1em}

        \item If $\mathcal{M}$ is a $\sigma$-algebra and $\mathcal{A} \subseteq \mathcal{M}$, then
        $\sigma_0(\mathcal{A}) \subseteq \mathcal{M}$.
    \end{enumerate}
\end{ftheorem}

\begin{proof}[Sketch of proof]
    Define $V = \{\mathcal{M} \subseteq \power(X): \mathcal{M} \text{ is } \sigma\text{-algebra}, \mathcal{A} \subseteq \mathcal{M}\}$.
    Notice that $V \neq \emptyset$ because $\power(X) \in V$. Then, define:

    $$\sigma_0(\mathcal{A}) = \bigcap_{\mathcal{M} \in V} \mathcal{M}$$

    Then, $\sigma_0(\mathcal{A})$ is a $\sigma$-algebra as it satisfies the properties of a $\sigma$-algebra, denoted
    in definition $2.1.1$.

\end{proof}

\begin{fremark}
    This is relevant. Often, to check that a $\sigma$-algebra has certain properties, 
    it is enough to check the property on a set of generators.
\end{fremark}

\section{Borel sets}

Take $(X, d)$ as a metric space, so that open sets are defined. Recall that:
$$\mathcal{T} = \{E \subseteq X: E \text{ is open}\}$$

\begin{fdefinition}
    The $\sigma$-algebra generated by $\mathcal{T}$ is called the \textbf{Borel $\sigma$-algebra} of $X$,
    denoted by:
    $$\B(X) := \sigma_0(\T)$$
    Any set $E \in \B(X)$ is a \textbf{Borel set}.
\end{fdefinition}

\begin{fremark}
    The following are Borel sets:

    \begin{itemize}
        \item Open sets
        \item Closed sets
        \item Countable intersections of open sets ($G_{\delta}$-sets)
        \item Countable unions of closed sets ($F_\sigma$-sets)
    \end{itemize}
\end{fremark}

\vspace{1em}

We will deal with:
$$X = \R = (-\infty, \infty)$$
but also:
$$X = \Rbar = [-\infty, \infty] = \R \cup \{-\infty, \infty\}$$

Let us define the arithmetic operations on $\Rbar$. Let $a \in \R$:

\begin{itemize}
    \item $a \pm \infty = \pm \infty$
    \item $a > 0: a \cdot \pm \infty = \pm \infty$
    \item $a < 0: a \cdot \pm \infty = \mp \infty$
    \item $a = 0: 0 \cdot \pm \infty = 0$
    \item $\infty - \infty$, $\infty/\infty$, $0/0$ are not defined.
\end{itemize}

Also, the open intervals in $\Rbar$ are the following:

\begin{itemize}
    \item $(a, b)$, with $a, b \in \R$
    \item $[-\infty, b)$
    \item $(a, \infty]$ 
\end{itemize}

\begin{fremark}
    We have that:
    \begin{align*}
        \B(\R) &:= \sigma_0(\{\text{open sets}\})\\
        &= \sigma_0(\{(a, b): a < b\})\\
        &= \sigma_0(\{[a, b]: a < b\})\\
        &= \sigma_0(\{(a, \infty): a \in \R\})
    \end{align*}
    \begin{align*}
        \B(\Rbar) &:= \sigma_0(\{\text{open sets}\})\\
        &= \sigma_0(\{(a, \infty]: a \in \R\})
    \end{align*}
    \begin{align*}
        \B(\R^N) &= \sigma_0(\{\text{open rectangles}\})
    \end{align*}
\end{fremark}

\section{Measures}

Let $(X, \M)$ be a measurable space.

\begin{fdefinition}
    A function $\mu: \M \to [0, \infty]$ is a (positive) \textbf{measure} on $\M$ if:
    \vspace{1em}
    \begin{enumerate}[label=(\roman*)]
        \item $\mu(\emptyset) = 0$
        \vspace{1em}
        \item $\{E_n\}_{n \in \N} \subseteq \M$, disjoint $\implies$ $\mu \left(\bigcup_{n \in \N} E_n\right) = \sum_{n \in \N} \mu(E_n)$
    \end{enumerate}
\end{fdefinition}

\begin{note}
    To avoid nonsenses, we always assume that $\exists E \in \M \;s.t.\; \mu(E) < \infty$
\end{note}

\underline{\textbf{Terminology:}} Let $X, \M, \mu$ defined as above:

\begin{itemize}
    \item $(X, \M, \mu)$ is a \textbf{measure space}.
    \item If $\mu(X) = 1$, then $(X, \M, \mu)$ is a \textbf{probability space} and $\mu$ is 
    a \textbf{probability measure}.
\end{itemize}

\begin{fdefinition}
    A measure $\mu$ is: 
    \vspace{1em}
    \begin{enumerate}
        \item \textbf{Finite} if $\mu(X) < \infty$
        \vspace{1em}
        \item \textbf{$\sigma$-finite} if $\exists \{E_n\}_{n \in \N} \subseteq \M$ s.t. 
        $$\mu(E_n) < \infty \;\; \forall n \in \N \quad \wedge \quad X = \bigcup_{n \in \N} E_n$$
    \end{enumerate}
\end{fdefinition}

\begin{fexample}
    Some examples of measures are:
    \vspace{1em}
    \begin{enumerate}
        \item (Trivial measure): For any $(X, \M)$, define $\mu$ as $\mu(E) = 0 \;\; \forall E \in \M$
        \vspace{1em}
        \item (Counting measure): For any $(X, \M)$, typically $\M = \power(X)$, define $\mu_{\#}$ as:
        $$\mu_{\#}(E) = \begin{cases}
            \#\{\text{elements of } E\} & E \text{ finite}\\
            \infty & \text{otherwise}
        \end{cases}$$
        \vspace{1em}
        \item (Dirac measure): For any $(X, \M)$, pick $x_0 \in X$. Then, define $\delta_{x_0}$ as:
        $$\delta_{x_0}(E) = \begin{cases}
            1 & x_0 \in E\\
            0 & x_0 \notin E
        \end{cases}$$
    \end{enumerate}
\end{fexample}

\subsection{Properties of measures}

\begin{ftheorem}[Basic properties]
    Let $(X, \M, \mu)$ be a measure space. Then:
    \vspace{1em}
    \begin{enumerate}[label=(\roman*)]
        \item $\mu$ is finitely additive: $E, F \in \M, E \cap F = \emptyset \implies \mu(E \cup F) = \mu(E) + \mu(F)$
        \vspace{1em}
        \item (Monotonicity): $E, F \in \M, E \subseteq F \implies \mu(E) \leq \mu(F)$
        \vspace{1em}
        \item (Excision property): $E, F \in \M, E \subseteq F, \mu(E) < \infty \implies \mu(F \setminus E) = \mu(F) - \mu(E)$
    \end{enumerate}
\end{ftheorem}

\begin{proof}
    The proof is straightforward:
    \begin{enumerate}[label=(\roman*)]
        \item Let $E, F \in \M, E \cap F = \emptyset$. Then:
        $$\mu(E \cup F) = \mu(E) + \mu(F)$$
        \begin{proof}
            Obvious, using $E_n = \emptyset$ for $n \geq 3$.
        \end{proof}
        \item Let $E, F \in \M, E \subseteq F$. Then:
        $$\mu(E) \leq \mu(F)$$
        \begin{proof}
            Let $F = E \cup (F \setminus E)$. Then:
            $$\mu(F) = \mu(E) + \mu(F \setminus E) \geq \mu(E)$$
        \end{proof}
        \item Let $E, F \in \M, E \subseteq F, \mu(E) < \infty$. Then:
        $$\mu(F \setminus E) = \mu(F) - \mu(E)$$
        \begin{proof}
            Let $F = E \cup (F \setminus E)$. Then:
            $$\mu(F) = \mu(E) + \mu(F \setminus E)$$
        \end{proof}
    \end{enumerate}

    This concludes the proof.

\end{proof}

\begin{ftheorem}[Continuity among monotone sequences]
    Let $(X, \M, \mu)$ be a measure space. Let $\{E_n\}_{n \in \N} \subseteq \M$ be a sequence of
    measurable sets. Then:
    \vspace{1em}
    \begin{enumerate}[label=(\roman*)]
        \item If $\{E_n\} \uparrow$, $E := \lim_{n} E_n = \bigcup_{n} E_n$, then:
        $$\mu(E) = \lim_{n} \mu(E_n)$$
        \item If $\{E_n\} \downarrow$, $E := \lim_{n} E_n = \bigcap_{n} E_n$, and
        $\mu(E_1) < \infty$, then:
        $$\mu(E) = \lim_{n} \mu(E_n)$$
    \end{enumerate}
    
\end{ftheorem}

\begin{proof}
The proof goes as follows:
\begin{enumerate}[label=(\roman*)]
    \item If $\mu(E_n) = \infty$ for some $n$, then the proof is trivial. 
    Otherwise, let $F_1 = E_1$ and $F_n = E_n \setminus E_{n-1}$ for $n \geq 2$.
    Then, we can check that:
    \begin{itemize}
        \item $F_n \in \M, \forall n \in \N$
        \item $\{F_n\}$ is a disjoint sequence.
        \item $E_n = \bigcup_{k=1}^{n} F_k$
        \item Because of the disjointness, we have that:
        $$\mu(E_n) = \sum_{k=1}^{n} \mu(F_k)$$
    \end{itemize}
    Then, we have that:
    \begin{align*}
    \mu(E) &= \mu\left(\lim_{n} E_n\right) = \mu \left( \bigcup_{n = 1}^{\infty} E_n \right) =\\
    &= \mu \left( \bigcup_{n = 1}^{\infty} F_n \right) = \sum_{n=1}^{\infty} \mu(F_n) =\\
    &=\sum_{n=1}^{\infty} \left( \mu(E_n) - \mu(E_{n-1}) \right) = \lim_{n} \mu(E_n)
    \end{align*}

    \item Define $G_n = E_1 \setminus E_n$. Then, check that:
    \begin{itemize}
        \item $G_n \in \M, \forall n \in \N$
        \item $\{G_n\} \uparrow$
    \end{itemize}

    By the previous result, we have that:
    $$\mu \left( \bigcup_{n=1}^{\infty} G_n \right) = \lim_{n} \mu(G_n)$$

    Then, on the right-hand side:
    \begin{align*}
        \lim_{n} \mu(G_n) &= \lim_{n} \mu(E_1 \setminus E_n) =\\
        &=\mu(E_1) - \lim_{n} \mu(E_n)
    \end{align*}

    On the left-hand side:
    \begin{align*}
        \mu \left( \bigcup_{n=1}^{\infty} G_n \right) &= \mu\left( \bigcup_{n=1}^{\infty} (E_1 \setminus E_n) \right) =\\
        &=\mu\left( E_1 \setminus \bigcap_{n=1}^{\infty} E_n \right) =\\
        &=\mu(E_1) - \mu(E)
    \end{align*}

    Finally, we have that:
    $$\mu(E_1) - \mu(E) = \mu(E_1) - \lim_{n} \mu(E_n)$$
    
    And because $\mu(E_1) < \infty$, we have that:
    $$\mu(E) = \lim_{n} \mu(E_n)$$
\end{enumerate}
\end{proof}

\begin{fremark}
    In (ii), the condition $\mu(E_1) < \infty$ is essential. Consider $(\N, \power(\N), \mu_{\#})$.
    Then, define the following sequence:
    $$E_n = \{n, n+1, n+2, \ldots\}$$
    Note that $E_n \subseteq E_{n-1}$. Also, note that for any $n \in \N$, we have that:
    $$\mu_{\#}(E_n) = \infty$$
    Then, we have that:
    $$\mu_{\#}\left( \bigcap_{n} E_n \right) = \mu_{\#}(\emptyset) = 0$$
    But:
    $$\lim_{n} \mu_{\#}(E_n) = \infty$$

    This shows that the condition $\mu(E_1) < \infty$ is essential.
\end{fremark}

\vspace{1em}

\begin{ftheorem}[$\sigma$-subadditivity]
    Let $(X, \M, \mu)$ be a measure space. Let $\{E_n\}_{n \in \N} \subseteq \M$ be a sequence of
    measurable sets. Then:
    $$\mu\left( \bigcup_{n} E_n \right) \leq \sum_{n} \mu(E_n)$$
\end{ftheorem}

\begin{proof}
    Let $F_1 = E_1$ and $F_n = E_n \setminus \left( \bigcup_{k=1}^{n-1} E_k \right)$ for $n \geq 2$.
    Then, we have that:
    \begin{itemize}
        \item $F_n \in \M, \forall n \in \N$
        \item $F_n \subseteq E_n, \forall n \in \N$
        \item $\{F_n\}$ is a disjoint sequence.
        \item $\bigcup_{n} E_n = \bigcup_{n} F_n$
    \end{itemize}

    Then, we have that:
    \begin{align*}
        \mu\left( \bigcup_{n} E_n \right) &= \mu\left( \bigcup_{n} F_n \right) =\\
        &=\sum_{n} \mu(F_n) \leq \sum_{n} \mu(E_n)
    \end{align*}

\end{proof}

\section{Sets of measure zero, negligible sets, complete measures}

\begin{fdefinition}
    Let $(X, \M, \mu)$ be a measure space. Then:
    \vspace{1em}
    \begin{enumerate}
        \item A set $E \in \M$ is a \textbf{set of measure zero} if $\mu(E) = 0$.
        \vspace{1em}
        \item A set $F \in X$ (not necessarily measurable) is a \textbf{negligible set} if $\exists E \in \M$ s.t.
        $F \subseteq E$ and $E$ is a set of measure zero.
    \end{enumerate}
\end{fdefinition}

\begin{fdefinition}
    Let $(X, \M, \mu)$ be a measure space. Then, we say that $\mu$ is a \textbf{complete measure}
    (alternatively, that $(X, \M, \mu)$ is a \textbf{complete measure space}) all negligible 
    sets are measurable.
\end{fdefinition}

\begin{fremark}[Completion of a measure space]
    A measure space $(X, \M, \mu)$ may not be complete. However, we can define the following:
    $$\overline{\M} = \{E \subseteq X: \exists F_1, F_2 \in \M: F_1 \subseteq E \subseteq F_2, \mu(F_2 \setminus F_1) = 0\}$$

    One can show that $\overline{\M}$ is a $\sigma$-algebra, and that 
    $\M \subseteq \overline{\M}$. Moreover, if $E, F_1, F_2$ are as above, define:
    $$\overline{\mu}(E) = \mu(F_1) = \mu(F_2)$$

    One can check that $(X, \overline{\M}, \overline{\mu})$ is a complete measure space.
\end{fremark}

\section{Towards the Lebesgue measure}

We would like to define a measure $\lambda$ with $X = \R$ (or $X = \R^N$) s.t. $\forall a < b$:
\begin{itemize}
    \item $\lambda((a, b)) = b - a$ \textbf{(length of the interval)}
    \item $\forall E, \lambda(E + x) = \lambda(E)$ \textbf{(translation invariance)}
\end{itemize}

In principle, we would like to define it in $\power(\R)$. Such a measure should satisfy $\lambda(\{a\}) = 0$.

\begin{ftheorem}[Ulam]
    The only measure on $\power(\R)$ that satisfies $\lambda(\{a\}) = 0 \; \forall a \in \R$ is
    the trivial measure.
\end{ftheorem}

\vspace{1em}

Therefore, we need to choose an $\M \subsetneq \power(\R)$. We can construct one as follows:

\begin{itemize}
    \item Starting family with a \say{measure}, e.g., $\T = \{(a, b): a < b\}$ and $f((a, b)) = b - a$.
    \item Construct an \say{outer measure} $\mu^*$ on $\power(\R)$.
    \item Restrict $\mu^*$ to a well-chosen $\sigma$-algebra $\M \subsetneq \power(\R)$.
\end{itemize}

\begin{fdefinition}
    Let $X$ be a set. An \textbf{outer measure} $\mu^{*}$ on $X$ is a function
    $$\mu^{*}: \power(X) \to [0, \infty]$$
    such that:
    \vspace{1em}
    \begin{enumerate}
        \item $\mu^{*}(\emptyset) = 0$
        \vspace{1em}
        \item (Monotonicity) $E \subseteq F \subseteq X \implies \mu^{*}(E) \leq \mu^{*}(F)$
        \vspace{1em}
        \item ($\sigma$-subadditivity) $\{E_n\}_{n \in \N} \subseteq \power(X) \implies \mu^{*}\left( \bigcup_{n \in \N} E_n \right) \leq \sum_{n \in \N} \mu^{*}(E_n)$
    \end{enumerate}
\end{fdefinition}

\vspace{1em}

\begin{fremark}
    Any measure $\mu$ is an outer measure. However, the converse is not true.
\end{fremark}

\vspace{1em}

\begin{fproposition}
    Let $\mathcal{E} \subseteq \power(X)$, $f: \mathcal{E} \to [0, \infty]$. Assume that 
    $\emptyset, X \in \mathcal{E}, f(\emptyset) = 0$. Then, $\forall E \subseteq X$ define:
    $$\mu^{*}(E) = \inf \left\{ \sum_{n=1}^{\infty} f(A_n): A_n \in \mathcal{E}, E \subseteq \bigcup_{n} A_n \right\}$$

    Then, $\mu^{*}$ is an outer measure.

\end{fproposition}

\begin{proof}
    The proof is omitted.

\end{proof}

\begin{fremark}
    In this generality, if $E \in \mathcal{E}$, then $f(E)$ and $\mu^{*}(E)$ may not be equal.
    We can only guarantee that $\mu^{*}(E) \leq f(E)$.
\end{fremark}

\vspace{1em}

\begin{fexample}
    There are some important examples:
    \vspace{1em}
    \begin{itemize}
        \item $X = \R$, $\mathcal{E} = \{(a, b): a,b \in \Rbar, a \leq b\}$
        $$f((a, b)) = \text{length}((a, b)) = b - a$$
        
        \vspace{1em}

        \item $X = \R^N$, $\mathcal{E} = \{(a_1, b_1) \times \ldots \times (a_N, b_N): a_i, b_i \in \Rbar, a_i \leq b_i\}$
        $$f((\underline{a}, \underline{b})) = \text{volume}((\underline{a}, \underline{b})) = \prod_{i=1}^{N} (b_i - a_i)$$
    \end{itemize}

    In both cases, the outer measure $\mu^{*}$ is called the \textbf{Lebesgue outer measure}. We
    will denote it by $\lambda^{*}$ (or $\lambda^{*}_N$ in the second case). Note that in this case,
    $\lambda^{*}(E) = f(E)$ for any $E \in \mathcal{E}$.
\end{fexample}

\begin{fremark}
    Any $\mu$ measure on $\power(X)$ is an outer measure. However, the converse is not true.
    In particular, $\exists A, B \subseteq \R$ s.t. $A \cap B = \emptyset$ and 
    $\lambda^{*}(A \cup B) < \lambda^{*}(A) + \lambda^{*}(B)$.
\end{fremark}

\subsection{Carathéodory's criterion}

\begin{fdefinition}[Carathéodory's condition]
    Let $\mu^{*}$ be an outer measure on $\power(X)$. A ser $E \subseteq X$ is 
    \textbf{$\mu^{*}$-measurable} if $\forall A \subseteq X$:
    $$\mu^{*}(A) = \mu^{*}(A \cap E) + \mu^{*}(A \cap E^c)$$
\end{fdefinition}

\vspace{1em}

\begin{flemma}[Equivalence of Carathéodory's condition]
    $E$ is $\mu^{*}$-measurable $\iff$ $\forall A \subseteq X$, $\mu^{*}(A) < \infty$:
    $$\mu^{*}(A) \geq \mu^{*}(A \cap E) + \mu^{*}(A \cap E^c)$$
\end{flemma}

\begin{proof}
    The proof is as follows:
    \begin{itemize}
        \item[($\Rightarrow$)]: Trivial
        \item[($\Leftarrow$)]: Let $A \subseteq X$, such that $\mu^{*}(A) < \infty$ and:
        $$\mu^{*}(A) \geq \mu^{*}(A \cap E) + \mu^{*}(A \cap E^c)$$

        Then, notice that $\{A \cap E, A \cap E^c\}$ is a covering of $A$. By subadditivity:
        $$\mu^{*}(A) \leq \mu^{*}(A \cap E) + \mu^{*}(A \cap E^c)$$

        Therefore, we have that:
        $$\mu^{*}(A) = \mu^{*}(A \cap E) + \mu^{*}(A \cap E^c)$$

        Meaning that $E$ is $\mu^{*}$-measurable. This concludes the proof.
    \end{itemize}
\end{proof}

\begin{ftheorem}[Carathéodory]
    Let $\mu^{*}$ be an outer measure on $\power(X)$. The family:
    $$\M = \{E \subseteq X: E \text{ is } \mu^{*}\text{-measurable}\}$$

    is a $\sigma$-algebra, and $\mu^{*}$ restricted to $\M$ (denoted $\mu = \mu^{*} |_{\M}$)
    is a complete measure.
\end{ftheorem}

\vspace{1em}

\begin{fremark}
    $(X, \M, \mu)$ as in the above theorem is sometimes called the \say{abstract Lebesgue measure space}.
    We will only prove the completeness of $\mu$.
\end{fremark}

\vspace{1em}

\begin{flemma}
    Let $(X, \M, \mu)$ be the measure space as in Carathéodory's theorem. Then, any $N \subseteq X$
    s.t. $\mu^{*}(N) = 0$ is $\mu$-measurable, i.e., $N \in \M$, and $\mu(N) = 0$.
\end{flemma}

\begin{proof}
    We have to show that $N$ satisfies Carathéodory's condition, or equivalently, that it
    satisfies the lemma $2.6.3$. Let $A \subseteq X$ be arbitrary. Then, notice that:
    $$\mu^{*}(A \cap N) \leq \mu^{*}(N) = 0$$

    Also, notice that:
    $$\mu^{*}(A \cap N^c) \leq \mu^{*}(A)$$

    Therefore, we have that:
    $$\mu^{*}(A \cap N) + \mu^{*}(A \cap N^c) \leq 0 + \mu^{*}(A) = \mu^{*}(A)$$

    By lemma $2.6.3$, we have that $N$ is $\mu^{*}$-measurable. By Carathéodory's theorem, we 
    have that $N$ is $\mu$-measurable. Finally, we have that $\mu(N) = \mu^{*}(N) = 0$.

\end{proof}

\begin{fcorollary}
    $\mu$ as in Carathéodory's theorem is a complete measure.
\end{fcorollary}

\begin{proof}
    Let $N \subseteq E$, and $\mu(E) = 0$ ($E \in \M$). Then, we have that:
    $$\mu(E) = 0 \implies \mu^{*}(E) = 0$$

    By monotonicity, we have that:
    $$\mu^{*}(N) \leq \mu^{*}(E) = 0$$

    Then, $\mu(N) = \mu^{*}(N) = 0$, thus $N \in \M$. This concludes the proof.

\end{proof}

\section{Lebesgue measure}

\begin{fdefinition}
    Let $\mathcal{E} = \{(a, b): a, b \in \Rbar, a \leq b\}$. Define:
    $$\lambda^{*}((a, b)) = b - a$$

    Then, $\lambda^{*}$ is the \textbf{Lebesgue outer measure} on $\R$.
\end{fdefinition}

\begin{ftheorem}
    Let $\lambda^{*}$ be the Lebesgue outer measure on $\mathcal{E} = \{(a, b): a, b \in \Rbar, a \leq b\}$.
    Then, by Carathéodory's theorem, the family:
    $$\Lcur(\R) = \{E \subseteq \R: E \text{ is } \lambda^{*}\text{-measurable}\}$$

    is a $\sigma$-algebra, called the \textbf{Lebesgue $\sigma$-algebra},
    and $\lambda^{*}$ restricted to $\Lcur(\R)$ (denoted $\lambda = \lambda^{*} |_{\Lcur(\R)}$)
    is a complete measure, called the \textbf{Lebesgue measure}. 
\end{ftheorem}

\begin{proof}
    The proof is omitted.

\end{proof}

\begin{fremark}
    The measure space $(\R, \Lcur(\R), \lambda)$ is called the \textbf{Lebesgue measure space}.
\end{fremark}

\begin{fproposition}
    Let $\lambda$ be the Lebesgue measure on $\R$. Then:
    \vspace{1em}

    \begin{enumerate}[label=(\roman*)]
        \item $a \in \R \implies \{a\} \in \Lcur(\R)$ and $\lambda(\{a\}) = 0$
        \vspace{1em}

        \item $E \subset \R$ at most countable $\implies E \in \Lcur(\R)$ and $\lambda(E) = 0$
    \end{enumerate}
\end{fproposition}

\begin{proof}
    The proof is as follows:
    \begin{enumerate}[label=(\roman*)]
        \item Let $a \in \R$. Then, we have that, for any $\varepsilon > 0$:
        $$E_1 = (a - \varepsilon, a + \varepsilon), \quad, E_2 = E_3 = ... = \emptyset$$

        is a covering of $\{a\}$. Then, by definition of $\lambda^{*}$:

        $$0 \leq \lambda^{*}(\{a\}) \leq \sum_{n=1}^{\infty} f(E_n) = 2 \varepsilon$$

        As $\varepsilon$ is arbitrary, we have that $\lambda^{*}(\{a\}) = 0$. By Lemma 2.6.5, we then 
        have that $\{a\} \in \Lcur(\R)$ and $\lambda(\{a\}) = 0$.

        \item Let $E \subseteq \R$ be at most countable. Then, we have that:
        $$E = \bigcup_{a \in E} \{a\}$$

        Because $\{a\} \in \Lcur(\R)$ and $\lambda(\{a\}) = 0$, we have that $E \in \Lcur(\R)$ and:

        $$\lambda(E) = \lambda\left( \bigcup_{a \in E} \{a\} \right) = \sum_{a \in E} \lambda(\{a\}) = 0$$

    \end{enumerate}
\end{proof}

\begin{fremark}
    We can point out 2 important consequences of the above proposition:
    \vspace{1em}

    \begin{enumerate}
        \item Arguing as in (i), we can show that the Lebesgue measure is translation invariant.
        That is, $\forall E \in \Lcur(\R)$, $\forall x \in \R$:
        $$\lambda(E + x) = \lambda(E)$$

        \item In particular, since $\Q$ is countable, we have that $\Q \in \Lcur(\R)$ and $\lambda(\Q) = 0$.
        In the measure sense, $\Q$ has very few elements with respect to $\R$. On the other hand,
        $\Q$ is dense in $\R$. In the topology sense, $\Q$ has a lot of points.

    \end{enumerate}
\end{fremark}

\vspace{1em}

\begin{fproposition}
    We have that: $\B(\R) \subset \Lcur(\R)$
\end{fproposition}

\begin{proof}
    Since $\B(\R) = \sigma_0(\{(a, \infty): a \in \R\})$, if we show that 
    $(a, \infty) \in \Lcur(\R), \quad \forall a \in \R$, then the prop. follows.\\

    Take $A \subset \R$, s.t. $\lambda^{*}(A) < \infty$. Then, we must show that:

    $$\lambda^{*}(A) \geq \lambda^{*}(A \cap (a, \infty)) + \lambda^{*}(A \cap (-\infty, a])$$

    Moreover, by a previous remark, one can assume that $a \notin A$. Then, take any countable
    covering of $A$ by open intervals:

    $$A \subseteq \bigcup_{n} I_n$$

    Then, let us define $A_{left} = A \cap (-\infty, a]$ and $I_{n, left} = I_n \cap (-\infty, a]$.
    Then, we notice that $\{I_{n, left}\}$ is a covering of $A_{left}$.\\

    In the same way, we define $A_{right} = A \cap (a, \infty)$ and $I_{n, right} = I_n \cap (a, \infty)$.
    Then, we notice that $\{I_{n, right}\}$ is a covering of $A_{right}$.\\

    Then, we have that:

    $$\lambda^{*}(A_{left}) \leq \sum_{n} \lambda^{*}(I_{n, left})$$
    $$\lambda^{*}(A_{right}) \leq \sum_{n} \lambda^{*}(I_{n, right})$$

    Summing both inequalities, we have that:

    $$\lambda^{*}(A_{left}) + \lambda^{*}(A_{right}) \leq \sum_{n} \lambda^{*}(I_{n, left}) + \sum_{n} \lambda^{*}(I_{n, right})$$
    $$ = \sum_{n} \lambda^{*}(I_n)$$

    Taking the infimum over all countable coverings of $A$, we have that:

    $$\lambda^{*}(A_{left}) + \lambda^{*}(A_{right}) \leq \lambda^{*}(A)$$

\end{proof}


\begin{fremark}
    In particular, we have that $\forall (a, b) \subset \R$:

    $$(a, b) \in \Lcur(\R), \quad \lambda((a, b)) = b - a$$
\end{fremark}

\vspace{1em}

We know that:
$$\B(\R) \subset \Lcur(\R) \subset \power(\R)$$

Are these inclusions strict? We already know that $\Lcur(\R) \neq \power(\R)$, by 
Ulam's theorem. In particular, $\exists E \subset \R$ not Lebesgue measurable 
(\textbf{Vitali sets}). More precisely: 

$$\forall E \in \Lcur(\R), \text{ s.t } \lambda(E) > 0, \exists V \subset E, \text{ s.t } V \notin \Lcur(\R)$$

The relation between $\B(\R)$ and $\Lcur(\R)$ is more subtle. It is clarified by the following
proposition:

\begin{fproposition}[Regularity of the Lebesgue measure]
    Let $E \in \R$. Then, the following are equivalent:
    \vspace{1em}
    \begin{enumerate}[label=(\roman*)]
        \item $E \in \B(\R)$
        \vspace{1em}
        \item $\forall \varepsilon > 0, \exists A \subset \R$ open set s.t.
        $$E \subset A \quad \text{and} \quad \lambda^{*}(A \setminus E) < \varepsilon$$

        \vspace{1em}    
    
        \item $\forall \varepsilon > 0, \exists G \subset \R$ of class $G_{\delta}$ s.t.
        $$E \subset G \quad \text{and} \quad \lambda^{*}(G \setminus E) = 0$$

        \vspace{1em}

        \item $\forall \varepsilon > 0, \exists C \subset \R$ closed set s.t.
        $$C \subset E \quad \text{and} \quad \lambda^{*}(E \setminus C) < \varepsilon$$

        \vspace{1em}

        \item $\forall \varepsilon > 0, \exists F \subset \R$ of class $F_{\sigma}$ s.t.
        $$F \subset E \quad \text{and} \quad \lambda^{*}(E \setminus F) = 0$$
    
    \end{enumerate}
\end{fproposition}

\vspace{1em}

We get as a consequence the following:

\begin{fcorollary}
    $\forall E \in \Lcur(\R)$, $\exists F, G \in \B(\R)$ s.t. $F \subset E \subset G$ and

    $$\lambda(G \setminus F) = \lambda(G \setminus E) + \lambda(E \setminus F) = 0$$

    In other words:
     $$\Lcur(\R) = \overline{\B(\R)}$$ 
     
    (But $\Lcur(\R) \neq \B(\R)$).

\end{fcorollary}

\begin{proof}[Proof. (Regularity of the Lebesgue measure)]
    The proof goes as follows:

    \begin{enumerate}
        \item[(i) $\Rightarrow$ (ii)]: \\
        
        Let $E \in \B(\R)$. Note that, since $A \in \Lcur(\R)$ for all $A$ open, we have that:

        $$\lambda^{*}(A \setminus E) = \lambda(A \setminus E)$$

        By definition of $\lambda^{*}$, we have that 
        $\forall \varepsilon > 0, \; \exists \{I_n\}_{n \in \N}$ s.t.

        $$E \subset \bigcup_{n} I_n \quad \text{and} \quad \sum_{n} \lambda(I_n) < \lambda^{*}(E) + \varepsilon$$

        Then, set $A = \bigcup_{n} I_n$. We have that $A$ is open, $E \subset A$ and:

        $$\lambda(A) \leq \sum_{n} \lambda(I_n) < \lambda(E) + \varepsilon$$
        $$\implies \lambda(A \setminus E) = \lambda(A) - \lambda(E)  < \varepsilon$$

        \item[(ii) $\Rightarrow$ (iii)]: \\
        
        Assume $\forall \varepsilon > 0, \; \exists A_\varepsilon$ open s.t. $E \subset A_\varepsilon$ and
        $\lambda(A_\varepsilon \setminus E) < \varepsilon$. Then, set $\varepsilon = 1/n, \; n \geq 1$ 
        (for ease of notation, $A_n = A_{1/n}$) and define:

        $$G = \bigcap_{n} A_{n}$$

        Then, $G$ is a $G_{\delta}$ set, $E \subset G$ and:

        $$ 0 \leq \lambda^{*}(G \setminus E) \leq \lambda^{*}(A_n \setminus E) < 1/n$$

        Taking the limit, we have that $\lambda(G \setminus E) = 0$.

        \item[(iii) $\Rightarrow$ (i)]: \\
        
        We know that $E \subset G, \; G \in \Lcur(\R)$ with $\lambda(G \setminus E) = 0$. 
        Then, we have that:

        $$ E = G \setminus (G \setminus E) \in \Lcur(\R)$$

        since $G \in \Lcur(\R)$ and $G \setminus E \in \Lcur(\R)$. The last is 
        because it is a negligible set and $\lambda$ is complete.

    \end{enumerate}
    
\end{proof}

\begin{fexample}[Cantor set]
    Let $T_0 = [0, 1]$. Then, construct $T_{n+1}$ from $T_n$ (recursively) by removing the
    inner third part of every interval in $T_n$:

    $$T_0 = [0, 1], $$
    $$T_1 = [0, 1/3] \cup [2/3, 1],$$
    $$T_2 = [0, 1/9] \cup [2/9, 1/3] \cup [2/3, 7/9] \cup [8/9, 1], \; ...$$


    Then, define the \textbf{Cantor set} as:

    $$C = \bigcap_{n} T_n$$

    It can be proven that:
    \begin{itemize}
        \item $C$ has the cardinality of $\R$
        \item $\lambda(C) = 0$
        \item $C$ is compact
        \item $C$ is nowhere dense (has no interior points), i.e., $\text{int}(C) = \emptyset$
        \item $\exists E \subset C$ s.t. $E \in \Lcur(\R)$ but $E \notin \B(\R)$
    \end{itemize}
    
\end{fexample}

