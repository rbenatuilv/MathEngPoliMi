\chapter{Introduction: Streaming data engineering}

In streaming data engineering, continuity of data analysis is a key concept. To process
large amounts of data, we have two main approaches:

\begin{itemize}
    \item Batch processing: this is the traditional approach where data is collected and stored, 
    then processed in a batch to obtain insights.
    \item Stream processing: in this approach, data is processed as it arrives, allowing for real-time
    insights.
\end{itemize}

In some cases, batch processing is enough, like when we need to process data that is not time-sensitive, such
as historical data. However, in many cases, we need to process data in real-time, like in fraud detection or
monitoring systems like fire alarms. In these cases, stream processing is the way to go. But, how do 
we enable this continuity of data flow?

\section{Enabling Continuity}

To enable continuity of data processing, we need four main components:

\begin{itemize}
    \item Sensors and actuators
    \item Connectivity
    \item Streaming data engineering
    \item Streaming data science
\end{itemize}

\subsection{Sensors and Actuators}

Sensors are devices that collect data from the environment, like temperature sensors, cameras, etc. Actuators
are devices that can act on the environment, like turning on a light, opening a door, etc. These devices
are the first step in the data processing pipeline. They are a way of giving the computers the means to gather 
data from the environment and act on it.\\

\textit{[...] sensor technology enable computers to observe, identify and understand the world - without
the limitations of human-entered data.} - \textbf{Kevin Ashton, the brander of "Internet of Things"}\\

\subsection{Connectivity}

This is the enabling and constraninig factor of the Internet of Things. It is the way that sensors and actuators
communicate with the rest of the system. This can be done through a variety of means, like WiFi, Bluetooth, etc.\\

In this case, the distance and the use case are the main factors that determine the type of connectivity that
we need. For example, if we need to connect a sensor to a computer that is far away, we might need to use a
cellular network, while if we need to connect a sensor to a computer that is close, we might use Bluetooth.\\

Also, bandwidth is a factor that we need to consider. If we need to send a lot of data, we might need a high
bandwidth connection, while if we need to send a small amount of data, we might use a low bandwidth connection.
Another important factor is the latency of the connection. If we need to send data in real-time,
we need a low latency connection, while if we don't need to send data in real-time, we might use a high latency
connection.\\

Note that most of the times, it comes down to a trade-off between these factors, so we need to choose the
right connectivity for our use case.

\subsection{Streaming Data Engineering}

This is the main topic of this chapter. This area focuses on the systems and infrastructure that we need to
manage various types of data streams efficiently. We have four main concepts:

\begin{itemize}
    \item Event-based systems: to tame myriads of tiny flows of data
    \item Data Stream Management Systems (DSMS): to handle continuos massive flows of unstoppable data
    \item Complex Event Processing (CEP): to manage continuos numerous flows of data that can turn into a torrent
    \item Event-driven architecture: to tame the forming of an inmense delta made of myriads of flows of any size and speed.
\end{itemize}

We will cover these concepts in more detail in the next sections.

\subsection{Streaming Data Science}

This is the area that focuses on the algorithms and techniques that we need to analyze data streams efficiently.
Note that changes in the data stream can happen at any time, and they cause ML models to lose accuracy. This is
why we need to use techniques to adapt to these changes. The three main components of this section are:

\begin{itemize}
    \item Time series analytics: to explain the past and forecast the future of a continuos
    flow of data without assuming data independence
    \item Streaming machine learning: to learn one data at a time from a continuos flow of data without
    assuming identically distributed data
    \item Continual learning: to do a long-life learning from a sequence of experiences without 
    forgetting past knowledge
\end{itemize}

This is the second topic of this course, and we will cover it in more detail in the next chapters.

\section{Importance of Streaming Data Analytics}

One idea: value is about time. Traditionally, data processing creates value by providing insights that
are generated with months or years of data gathering. However, in many cases, we need to process data in real-time
to create value. \\

Continuos streaming analysis creates value by providing insights that are generated within seconds
or minutes, allowing for real-time decision making, in other words, reactive applications that can respond to
events as they happen.\\

Streaming data analytics is important, as it can:

\begin{itemize}
    \item Provide real-time insights
    \item Reduce costs
    \item Improve efficiency
    \item Create innovative products
    \item Generate new revenue streams
\end{itemize}

