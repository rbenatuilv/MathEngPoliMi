\chapter{Compact operators}

\begin{note}
    We will work with $X, Y$ Banach spaces.
\end{note}

\begin{fdefinition}
    Let $K: X \to Y$ be a linear operator. We say that $K$ is \textbf{compact} if:
    $$\forall E \subset X \text{ bounded }, K(E) \text{ is relatively compact, i.e., } \overline{K(E)} \text{ is compact}$$

    Or, equivalently:
    $$\forall \{x_n\}_n \subset X \text{ bounded, } \exists \{K(x_{n_k})\}_k \subset Y \text{ (strongly) convergent subsequence}$$

\end{fdefinition}

\vspace{1em}

\begin{fproposition}
    Let $K: X \to Y$ be a linear compact operator. Then $K$ is bounded, i.e., $K \in \Lcur(X, Y)$.
\end{fproposition}

\begin{proof}
    We know that $B_1(0) \subset X$ is bounded. Then, $\overline{K(B_1(0))}$ is compact in $Y$.
    Therefore, $\overline{K(B_1(0))}$ is bounded in $Y$.

    Then, $\exists M > 0$ such that $\norm{K(x)} \leq M \quad \forall x \in B_1(0)$.
\end{proof}

\begin{fremark}
    The above property is not true for non-linear compact operators.
\end{fremark}

\begin{fexercise}[Compactness of the integral map]
    Let $K: C([0, 1]) \to C([0, 1])$ be the integral map:

    $$K(f)(x) = \int_0^x f(t) \; dt \quad \forall x \in [0, 1]$$

    and note that it is linear. Prove that $K$ is compact.\\
    
    (Hint: take $\{u_n\}_n \subset C([0,1])$ bounded, and prove that $\{K(u_n)\}_n$ has a convergent subsequence
    using the Arzelà-Ascoli theorem).
\end{fexercise}

\vspace{1em}

\begin{fdefinition}
    We say that $T \in \Lcur(X, Y)$ is a \textbf{finite rank operator} if:
    $$ dim \; R(T) < \infty$$

    (Note: $R(T) = T(X)$).
\end{fdefinition}

\begin{example}
    As many as you want:\\

    $T \in X^*$: $C^k([a, b]) \to \prob^k$ polynomials of degree $k$:
    \begin{itemize}
        \item Taylor expansion
        \item Lagrange interpolation
        \item etc.
    \end{itemize}
\end{example}

\begin{fproposition}
    Let $T \in \Lcur(X, Y)$ be a finite rank operator. Then $T$ is compact.
\end{fproposition}

\begin{proof}
    Let $A \subset X$ be bounded. Then, $T(A)$ is bounded in $Y$, and $\overline{T(A)}$ is bounded
    and closed in $Y$.\\

    Since $dim \; R(T) < \infty$, $\overline{T(A)}$ is compact in $Y$.
\end{proof}

\begin{fdefinition}
    We denote $\K(X, Y)$ as the set of compact operators from $X$ to $Y$, i.e.:

    $$\K(X, Y) = \{K \in \Lcur(X, Y) \; | \; K \text{ is compact}\}$$
\end{fdefinition}

\vspace{1em}

\begin{ftheorem}
    Let $X, Y$ be Banach spaces. Then $\K(X, Y)$ is a closed vector subspace of $\Lcur(X, Y)$.
\end{ftheorem}

\begin{fremark}
    Now, to check that $T$ is compact, it is enough to find a sequence $\{T_n\}_n \subset \K(X, Y)$
    such that $T_n \to T$ in $\Lcur(X, Y)$, i.e., $\norm{T_n - T}_{\Lcur(X, Y)} \to 0$.
\end{fremark}

\vspace{1em}

\begin{ftheorem}[Compact operators vs weak convergence]
    Let $X, Y$ be Banach, then:
    \vspace{1em}
    \begin{enumerate}[label=(\roman*)]
        \item If $T \in \K(X, Y)$, then:
        $$\{x_n\}_n \subset X \text{ s.t. } x_n \weakto x \text{ in } X \implies T(x_n) \to T(x) \text{ in } Y$$

        \vspace{1em}

        \item If $X$ is reflexive, then, the converse is also true, i.e., $T \in \K(X, Y)$ if\\
        $\forall \{x_n\}_n \subset X$:
        $$x_n \weakto x \text{ in } X \implies T(x_n) \to T(x) \text{ in } Y$$
    
    \end{enumerate}
    
\end{ftheorem}

\vspace{1em}

\begin{fproposition}
    Let $T \in \K(X, Y)$, and $dim Y = \infty$. Then, $T$ cannot be surjective.
\end{fproposition}

\begin{fproposition}
    Take either $T \in \Lcur(X, Y)$, $S \in \K(X, Y)$ or $T \in K(X, Y)$ and $S \in \Lcur(X, Y)$. Then:
    $$ S \circ T \in \K(X, Y)$$
\end{fproposition}

\begin{proof}
    Trivial, because bounded operators map bounded sets to bounded sets, and precompact sets to precompact sets.

\end{proof}

