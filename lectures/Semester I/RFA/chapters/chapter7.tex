\chapter{Derivatives of measures}

Let $(X, \M, \mu)$ be a complete measure space. We know
that, given $\Phi: X \to [0, \infty]$ measurable, the 
function:

$$\nu_{\Phi}(E) := \int_E \Phi d\mu = \int_E d \nu_{\Phi}$$

is a measure on $(X, \M)$. Given $\mu, \nu$ measures on 
$(X, \M)$, is it true that there exists $\Phi$ such that

$$\nu(E) = \int_E \Phi d\mu \quad \forall E \in \M$$

We will study this question in this chapter.

\begin{fdefinition}
    Let $\mu, \nu$ measures on $(X, \M)$. If $\exists \Phi$ s.t 

    $$\nu(E) = \int_E \Phi d\mu \quad \forall E \in \M$$

    then $\Phi$ is the \textbf{Radon-Nikodym derivative} of $\nu$
    with respect to $\mu$ and we write:
    $$\Phi = \frac{d\nu}{d\mu}$$
\end{fdefinition}

\begin{fdefinition}
    Let $\mu, \nu$ measures on $(X, \M)$. Then $\nu$ is \textbf{absolutely 
    continuous} with respect to $\mu$ (\say{$\nu << \mu$}) if:

    $$\forall E \in \M, \quad \mu(E) = 0 \Rightarrow \nu(E) = 0$$
\end{fdefinition}

\begin{flemma}[Necessary condition]
    Let $\mu, \nu$ measures on $(X, \M)$. If $\nu$ has a Radon-Nikodym 
    derivative with respect to $\mu$, then $\nu$ is absolutely continuous 
    with respect to $\mu$.
\end{flemma}

\begin{proof}
    Assume $\nu$ has a Radon-Nikodym derivative with respect to $\mu$.
    Then:

    $$\nu(E) = \int_{E} \Phi d\mu = 0$$
\end{proof}

\begin{fexercise}
    Take $(X, \M) = (\R , \Lcur(\R))$, $\mu = \lambda$ the Lebesgue
    measure and $\nu = \delta_0$ the Dirac measure at $0$. Show that

    $$\nexists \frac{d\nu}{d\mu}$$
\end{fexercise}

\vspace{1em}

\section{The Radon-Nikodym Theorem}

\begin{ftheorem}[Radon-Nikodym Theorem]
    Let $(X, \M)$ be a measurable space, $\mu, \nu$ measures and 
    $\mu$ is $\sigma$-finite. Then:

    $$\nu << \mu \iff \exists \frac{d\nu}{d\mu}$$
\end{ftheorem}

\begin{fcorollary}
    Let $\nu$ be a measure on $(\R^N, \Lcur(\R^N))$ and $\mu << \lambda$.
    Then:

    $$\exists \Phi := \frac{d\nu}{d\mu}: \quad \nu(E) = \int_E \Phi \, d\lambda \quad \forall E \in \Lcur(\R^N)$$

    (Indeed, $\lambda$ is $\sigma$-finite)
\end{fcorollary}






