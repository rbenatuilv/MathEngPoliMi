\chapter{Compactness, Density and Separability}

\section{Compactness}

We say that $(X, d)$ is a metric space.\\

\begin{fdefinition}
    $E \subset X$ is \textbf{compact} if from any open covering 
    $\{A_i\}_{i \in I}$ ($A_i$ open $\forall i \in I$, $E \subset \bigcup_{i \in I} A_i$)
    we can extract a finite subcovering.\\

    Typically, we define it as follows:\\

    Take $E$, fix $r > 0$ and consider $\{B_r(x)\}_{x \in E}$, the open balls of radius $r$ centered at $x \in E$.\\
    Then, $E$ is compact if there exists $x_1, ..., x_k \in E$ s.t.

    $$E \subset \bigcup_{i=1}^k B_r(x_i)$$

\end{fdefinition}

\vspace{1em}

\begin{fdefinition}
    $E$ is \textbf{sequentially compact} if $\forall \{x_n\}_{n \in \N} \subset E$, there
    exists a subsequence $\{x_{n_k}\}_{k \in \N}$ that converges to some $x \in E$.\\
\end{fdefinition}

\begin{fremark}
    The two definitions are equivalent in metric spaces.
\end{fremark}

\vspace{1em}

\begin{fdefinition}
    $E \subset X$ is \textbf{relatively compact} if $\overline{E}$ is compact.
\end{fdefinition}

\vspace{1em}

\begin{ftheorem}[Heine-Borel]
    Let $(X, \norm{\cdot})$ be a normed vector space. If 
    $dim(X) < \infty$, then $E \subset X$ is compact $\iff$ $E$ is closed and bounded.
\end{ftheorem}

\begin{fremark}
    The theorem is not true in infinite-dimensional spaces. In particular,
    if $E \subset X$ is compact, then $E$ is closed and bounded, but the converse
    is not true.
\end{fremark}

\vspace{1em}

\begin{ftheorem}[Riesz]
    Let $(X, \norm{\cdot})$ be a normed vector space. Then:

    $$\overline{B_1(0)} \text{ is compact} \iff dim(X) < \infty$$
\end{ftheorem}

\begin{proof}
    \begin{itemize}
        \item[($\Leftarrow$)] Exercise.
        \item[($\Rightarrow$)] Suppose $\overline{B_1(0)} = \{x \in X: \norm{x} \leq 1\}$ is compact.\\
        
        Consider $\{ B_{1/2}(x) \}_{x \in \overline{B_1(0)}}$.
        Then:
        $$\overline{B_1(0)} \subset \bigcup_{x \in \overline{B_1(0)}} B_{1/2}(x)$$

        By compactness, $\exists x_1, ..., x_k \in \overline{B_1(0)}$ s.t.
        $$\overline{B_1(0)} \subset \bigcup_{i=1}^k B_{1/2}(x_i)$$
        $$\subset \bigcup_{i=1}^k \overline{B_{1/2}(x_i)}$$

        This means that $\forall x \in \overline{B_1(0)}$, $\exists i \in \{1, ..., k\}$, s.t.

        $$x = x_i + z \text{ for some } \norm{z} \leq 1/2$$

        Define $V = span\{x_1, ..., x_k\}$. Then, $V \subset X$ is a vector subspace
        and $dim V \leq k < \infty$.\\
        
        We can then rewrite the previous implication as: $\forall x \in \overline{B_1(0)}$, $\exists v \in V$ s.t.

        $$x = v + z \text{ for some } \norm{z} \leq 1/2$$

        Now, take $y \in X$, s.t. $y \neq 0$. Then, notice that:

        $$\frac{y}{\norm{y}} \in \overline{B_1(0)}$$

        So there exists $v \in V$ and $z: \norm{z} \leq 1/2$ s.t.

        $$\frac{y}{\norm{y}} = v + z$$

        Then, $y = \norm{y}v + \norm{y}z$. We rewrite this as:

        $$y = v' + z'$$

        where $v' = \norm{y}v \in V$ and $\norm{z'} \leq \norm{y}/2$.\\

        Then, take any $x \in X$ and apply the previous result to $y = x$:

        $$x = v_1 + z_1, \quad v_1 \in V, \quad \norm{z_1} \leq \norm{x}/2$$

        Then, apply it again to $y = z_1$:

        $$x = v_1 + v' + z_2, \quad v_1, v' \in V, \quad \norm{z_2} \leq \norm{z_1}/2 \leq \norm{x}/4$$

        Notice that, because $V$ is a vector space, $v_1 + v' \in V$. Then, we rewrite the previous equation as:

        $$x = v_2 + z_2, \quad v_2 \in V, \quad \norm{z_2} \leq \norm{x}/4$$

        By induction:

        $$x = v_n + z_n, \quad v_n \in V, \quad \norm{z_n} \leq \norm{x}/2^n$$

        Notice that $z_n \to 0$ as $n \to \infty$. Then:

        $$v_n = x - z_n \to x \text{ as } n \to \infty$$

        Meaning that the sequence $\{v_n\}_n \subset V$ converges to $x \in X$, and because
        $V$ is a finite-dimentional vector subspace, it is closed, so $x \in V$.\\

        With this, we have shown that $X = V$, and therefore, $dim X \leq k < \infty$.

    \end{itemize}
\end{proof}

\section{Compactness in $C([a, b])$}

\begin{note}
    We always deal with $(C([a, b]), \norm{\cdot}_{\infty})$, which is Banach
\end{note}

\begin{fdefinition}
    Let $\{u_n\}_n \subset C([a, b])$ a sequence of continuous functions. 
    Then, we say that it is \textbf{uniformly equicontinuous} if $\forall \varepsilon > 0$,
    $\exists \delta > 0$ s.t.,

    $$|x - y| < \delta \implies |u_n(x) - u_n(y)| < \varepsilon, \quad \forall 
    x, y \in [a, b], \forall n \in \N$$

    (The value of $\delta$ only depends on $\varepsilon$)
\end{fdefinition}

\vspace{1em}

\begin{ftheorem}[Ascoli-Arzelà]
    Take $\{u_n\}_n \subset C([a, b])$. Assume that:
    \vspace{1em}
    \begin{enumerate}[label=(\roman*)]
        \item $\{u_n\}_{n \in \N}$ is uniformly bounded, i.e.:
        $$\exists \, 0 < M < \infty, \quad \norm{u_n}_{\infty} \leq M \quad \forall n \in \N$$

        \vspace{1em}

        \item $\{u_n\}_n$ is uniformly equicontinuous.
    \end{enumerate}
    \vspace{1em}

    Then, there exists a subsequence $\{u_{n_k}\}_{k \in \N}$ and $u \in C([a, b])$
    s.t. $u_{n_k} \to u$ as $k \to \infty$
\end{ftheorem}

\begin{fexample}
    Let $\{u_n\}_n \subset C^1([a, b]) \subset C([a, b])$. Assume that:
    \vspace{1em}
    \begin{enumerate}
        \item $\norm{u_n} \leq M. \; \forall n$
        \vspace{1em}
        \item $\norm{u'_n}_n \leq L, \; \forall n$ 
    \end{enumerate}
    \vspace{1em}

    Then, the theorem applies. Indeed: $1) \implies (i)$ in Ascoli-Arzelà.
    To check equicontinuity: $\forall x, y \in [a, b], x \neq y$:

    $$|u_n(x) - u_n(y)| = |u'_n(\zeta) \cdot (x - y)| \quad (\text{Mean Value Thm.})$$ 

    $$\implies |u_n(x) - u_n(y)| \leq |u'_n(\zeta)| \cdot |x - y|$$
    $$\leq \norm{u'_n}_{\infty} \cdot |x - y|$$
    $$\leq L |x - y|, \quad \forall n \in \N$$

    $$\implies \text{ equicontinuity (take } \delta = \frac{\varepsilon}{L})$$

    Roughly, the thm. implies that \say{boundedness in $C^1 \implies$ compactness in $C^0$ }.
\end{fexample}

\vspace{1em}

\begin{fremark}
    The same is true for Lipschitz continuos functions with uniformly bounded Lipschitz constant.\\

    Also, there are similar theorems in $L^p$ with:
    $$W^{1, p} = \{L^p \text{ functions having } L^p \text{ weak derivatives}\}$$
    and \say{boundedness in $W^{1, p} \implies$ compactness in $L^p$ }.
\end{fremark}

\section{Density, separability}

\begin{fdefinition}
    We say that $D \subset X$ is \textbf{dense} if $\overline{D} = X$, i.e.:

    $$\forall x \in X, \; \exists \{y_n\}_n \subset D: \; y_n \to x \in X$$
\end{fdefinition}

\vspace{1em}

\begin{fdefinition}
    $X$ is \textbf{separable} if $\exists D \subset X$, s.t. $D$ is countable and dense
\end{fdefinition}

\begin{fremark}
    Typically, one uses dense subsets because \say{continuous properties, true on $D$, are also true on $X$}.
    When $D$ is separable, you have few elements to check the property. 
\end{fremark}

\begin{example}
    $\R, \R^N, \Omega \subset \R^N$ are all separable: $\overline{\Q} = \R$ and $\Q$ is countable.
\end{example}

\begin{ftheorem}
    The following spaces are separable:
    \vspace{1em}
    \begin{itemize}
        \item $(C([a, b]), \norm{\cdot}_{\infty})$
        \vspace{1em}
        \item $(L^p(\R), \norm{\cdot}_p)$ for $1 \leq p < \infty$
    \end{itemize}
    \vspace{1em}
    and $(L^{\infty}(\R), \norm{\cdot}_{\infty})$ is \textbf{NOT} separable.
\end{ftheorem}

\subsection{Dense subspaces}

For continuous functions, we have the following result:\\

\begin{ftheorem}[Stone-Weierstrass]
    Polynomials are dense in $C([a, b])$, i.e.:

    $$\forall f \in C([a, b]), \; \forall \varepsilon > 0, \; \exists P(x) \text{ polynomial s.t.}$$
    $$\norm{f - P}_{\infty} < \varepsilon$$

    Note that polynomials with coefficients in $\Q$ are countable.
\end{ftheorem}
\vspace{1em}

For $L^p$ spaces, we have the following dense subspaces:
\begin{itemize}
    \item Simple functions
    \item Continuous (or more regular) functions
\end{itemize}

\begin{note}[Recall]
    $s: \R \to \R$ is (measurable and) simple if:
    $$s = \sum_{i=1}^k \alpha_i \1_{A_i}$$

    where $\alpha_i \in \R$ and $A_i \in \Lcur(\R)$ are disjoint sets, s.t.:
    $$\bigcup_{i=1}^k A_i = \R$$

    We know that $s$ simple $\implies s \in L^{\infty}(\R)$. Does $s$ simple $\implies s \in L^p(\R)$?
    For $p \in [1, \infty)$, we have that:
    $$s \in L^p(\R) \iff \lambda(\{x: s(x) \neq 0\}) < \infty$$

\end{note}

\begin{fdefinition}
    We define $\tilde{\rho}(\R)$ as the set of simple functions on $\R$,
    such that $\lambda(\{x: s(x) \neq 0\}) < \infty$:

    $$\tilde{\rho}(\R) = \{s: \R \to \R \text{ simple} \; | \; \lambda(\{x: s(x) \neq 0\}) < \infty\}$$
\end{fdefinition}

\begin{ftheorem}
    $\tilde{\rho}(\R)$ is dense in $L^p(\R)$ for $1 \leq p < \infty$.
\end{ftheorem}

\vspace{1em}

\begin{fdefinition}
    We define the following concepts:
    \vspace{1em}
    \begin{enumerate}
        \item $u: \R \to \R$. The \textbf{support} of $u$ is defined as:
        $$supp(u) = \overline{\{x: u(x) \neq 0\}}$$
        
        \item $C_c(\R) = \{u \in C(\R): supp(u) \text{ is compact}\}$
        \vspace{1em}

        \item $C_c^{\infty}(\R) = \{u \in C_c(\R): u \text{ is infinitely differentiable}\} = \C_0^{\infty}(\R) = \mathcal{D}(\R)$
    \end{enumerate}
\end{fdefinition}

\begin{ftheorem}
    $C_c^{\infty}(\R)$ is dense in $L^p(\R)$ for $1 \leq p < \infty$.
\end{ftheorem}

\begin{fcorollary}
    $C_c(\R)$ is dense in $L^p(\R)$ for $1 \leq p < \infty$.\\

    ($D \subset X$ dense, $D \subset E \subset X \implies E$ dense in $X$)
\end{fcorollary}

\vspace{1em}

\begin{fremark}
    $C_c^(\R)$ is not dense in $L^{\infty}(\R)$. Indeed, take

    $$\mathcal{H}(x) = \begin{cases}
        1 & \text{if } x > 0 \\
        0 & \text{otherwise}
    \end{cases}
    $$

    Then, $\mathcal{H} \in L^{\infty}(\R)$, but now suppose that we have 
    a function $g \in C_c(\R)$ s.t.: 
    
    $$\norm{\mathcal{H} - g}_{\infty} \leq 1/3 $$

    Then:
    $$|\mathcal{H}(x) - g(x)| \leq 1/3, \quad \text{a.e. } x \in \R$$
    $$\implies \mathcal{H}(x) - 1/3 \leq g(x) \leq \mathcal{H}(x) + 1/3$$

    This implies that $g$ cannot be continuous in $x = 0$. Conttradiction.
\end{fremark}

\vspace{1em}

\begin{note}
    Let us see that $L^{\infty}(\R)$ is not separable.\\
\end{note}

\begin{flemma}
    Take $X$ Banach. Assume that $\{A_i\}_{i \in I}$ is s.t.:
    \vspace{1em}
    \begin{enumerate}[label=(\alph*)]
        \item $\forall i \in I, \; A_i \subset X$ is open and non-empty
        \vspace{1em}
        \item $\forall i \neq j \in I, \; A_i \cap A_j = \emptyset$
        \vspace{1em}
        \item $I$ is more than countable.
    \end{enumerate}
    \vspace{1em}

    Then, $X$ is not separable.
\end{flemma}

\begin{proof}
    By contradiction. Assume that $X$ is separable. Then, $\exists \{x_n\}_{n \in \N} \subset X$ s.t.:
    $$X = \overline{\bigcup_{n \in \N} \{x_n\}}$$

    Then, $\forall A_i, \; \exists x_{n_i} \in A_i$. 
    This is because $A_i$ is non-empty, then $\exists z_i \in A_i$, and 
    because $\{x_n\}_n$ dense, $\exists \{x_{n_k}\}_{k \in \N}$ s.t. $x_{n_k} \to z_i$ as $k \to \infty$.
    Notice that $A_i \subset X$ is open, so the sequence $\{x_{n_k}\}_k$ is eventually in $A_i$.\\
    
    Since $A_i \cap A_j = \emptyset$,
    $x_{n_i} \neq x_{n_j}$, i.e., $n_i \neq n_j$.\\

    Then, we have a map $i \to n_i$ that is injective, so $I$ is at most countable. Contradiction.

\end{proof}

\begin{ftheorem}
    $L^{\infty}(\R)$ is not separable.
\end{ftheorem}

\begin{proof}
    We use the previous lemma. $\forall \alpha \in \R^+ = (0, \infty)$, we define:

    $$g_{\alpha}(x) = \chi_{[-\alpha, \alpha]}(x) = \begin{cases}
        1 & \text{if } |x| \leq \alpha \\
        0 & \text{otherwise}
    \end{cases}$$

    Notice that, if $\alpha_1 \neq \alpha_2$, then $\norm{g_{\alpha_1} - g_{\alpha_2}}_{\infty} = 1$.\\
    $$\implies B_{1/2}(g_{\alpha_1}) \cap B_{1/2}(g_{\alpha_2}) = \emptyset$$

    Indeed, $\forall f \in L^{\infty}(\R)$, we have that:

    $$1 = \norm{g_{\alpha_1} - g_{\alpha_2}}_{\infty} \leq \norm{g_{\alpha_1} - f}_{\infty} + \norm{f - g_{\alpha_2}}_{\infty}$$
    $$\implies \text{ at least one of the norms is greater than } 1/2$$

    Then, we have a family of open sets $\{B_{1/2}(g_{\alpha})\}_{\alpha \in \R^+}$ that satisfies the conditions of the lemma.\\

    Then, $L^{\infty}(\R)$ is not separable.
    
\end{proof}

