\chapter{Measurable functions}

\begin{fdefinition}
    Given $f: X \to Y$, it is well-defined the \textbf{preimage} (or 
    counterimage) of $f$ as:

    $$f^{-1}: \power(Y) \to \power(X), \quad f^{-1}(E) = \{x \in X: f(x) \in E\}$$
\end{fdefinition}

\begin{fremark}
    Preimages work well with set operations:
    \begin{itemize}
        \item $f^{-1}(E \cup F) = f^{-1}(E) \cup f^{-1}(F)$
        \item $f^{-1}(E \cap F) = f^{-1}(E) \cap f^{-1}(F)$
        \item $f^{-1}(E \setminus F) = f^{-1}(E) \setminus f^{-1}(F)$
    \end{itemize}

    Note this is not true for images.
\end{fremark}

\vspace{1em}

\begin{fdefinition}
    Let $(X, \M)$ and $(Y, \Ncur)$ be measurable spaces. A function $f: X \to Y$ is 
    \textbf{measurable} if $\forall E \in \Ncur$, we have that $f^{-1}(E) \in \M$.
    We also say that $f$ is \textbf{($\M$,$\Ncur$)-measurable}.
\end{fdefinition}

\vspace{1em}

\begin{fproposition}
    Let $(X, \M)$ and $(Y, \Ncur)$ be measurable spaces, and $\rho \subset \Ncur$ s.t.
    $\Ncur = \sigma_0(\rho)$. Then, $f: X \to Y$ is measurable $\iff$ $\forall E \in \rho$,
    we have that $f^{-1}(E) \in \M$.
\end{fproposition}

\begin{proof}
    The proofs goes as follows:

    \begin{itemize}
        \item[($\Rightarrow$)]: Trivial
        \item[($\Leftarrow$)]:
        Define $\Sigma := \{E \subset Y: f^{-1}(E) \in \M\}$. We have:

        \begin{itemize}[label=$\bullet$]
            \item $\rho \subset \Sigma$ as a consecuence of ($\forall E \in \rho, f^{-1}(E) \in \M$)
            \item $\Sigma$ is a $\sigma$-algebra (check as an exercise)
        \end{itemize}

        Then, we have that $\Ncur = \sigma_0(\rho) \subset \Sigma$. Therefore, $f$ is measurable. 

    \end{itemize}
\end{proof}

\begin{fdefinition}
    Suppose that $\M \supseteq \B(X)$ and $\Ncur = \B(Y)$.
    We say that $f: X \to Y$ is:
    \vspace{1em}

    \begin{itemize}
        \item \textbf{Borel measurable} if $f$ is ($\B(X),\B(Y)$)-measurable.
        \vspace{1em}
        \item \textbf{Lebesgue measurable} if it is $(\M, \B(Y))$-measurable.
    \end{itemize}
\end{fdefinition}

\begin{fremark}
    If $f: X \to Y$ is Borel measurable, then it is Lebesgue measurable. 
    The converse is not true.\\

    Also, we never deal with $\Lcur(Y)$.
\end{fremark}

\vspace{1em}

\begin{fcorollary}
    $f$ is Borel measurable $\iff$ $f^{-1}(E) \in \B(X), \; \forall E \in Y$ open.
    Also, $f$ is Lebesgue measurable $\iff$ $f^{-1}(E) \in \M, \; \forall E \in Y$ open.
\end{fcorollary}

\begin{proof}
    It follows from the previous proposition, since $\B(Y) = \sigma_0(\{E \subset Y: E \text{ open}\})$.
\end{proof}

\begin{fdefinition}
    We say that $f$ is \textbf{continuous} $\iff$ $f^{-1}(E) \subset X$ is open $\forall E \subset Y$ open.
\end{fdefinition}

\begin{fproposition}
    If $f: X \to Y$ is continuous, then $f$ is Borel measurable (and thus Lebesgue measurable).
\end{fproposition}

\begin{proof}
    Let $E \subset Y$ be open. By continuity of $f$, we have that $f^{-1}(E)$ is open.
    Then $f^{-1}(E) \in \B(X)$, and thus $f$ is Borel measurable.\\

    Note that the proposition is false when $\Ncur \supsetneq \B(Y)$.
\end{proof}

\section{Operations on measurable functions}

\begin{fproposition}
    Let $f: X \to Y$ be Lebesgue measurable, and $g: Y \to Z$ be continuous. Then:

    $$g \circ f: X \to Z \text{ is Lebesgue measurable}$$
\end{fproposition}

\begin{fcorollary}
    Let $f: X \to Y$ be Lebesgue measurable. Then:
    \vspace{1em}
    \begin{itemize}
        \item $f^+(x) = \max\{f(x), 0\}$ is Lebesgue measurable
        \vspace{1em}
        \item $f^-(x) = \max\{-f(x), 0\}$ is Lebesgue measurable
        \vspace{1em}
        \item $|f(x)| = f^+(x) + f^-(x)$ is Lebesgue measurable
    \end{itemize}

\end{fcorollary}

\begin{proof}
    Let $f$ be Lebesgue measurable, and $g: \R \to \R$ be continuous. Then, take
    $E \subset Z$ open. We have that:

    $$(g \circ f)^{-1}(E) = f^{-1}(g^{-1}(E))$$

    Since $g$ is continuous, $g^{-1}(E)$ is open. Then, $f^{-1}(g^{-1}(E)) \in \M$
\end{proof}

\begin{fproposition}
    Let $f, g: X \to \R$ be Lebesgue measurable, and $\Phi: \R^2 \to \R$ be continuous.
    Then, $h(x) = \Phi(f(x), g(x))$ is Lebesgue measurable. 
\end{fproposition}

\begin{proof}
    Let $h(x) = \Phi(f(x), g(x)) = (\Phi \circ \Psi)(x)$, where $\Psi: X \to \R^2$ is defined as

    $$\Psi(x) = (f(x), g(x))$$

    Then, we should prove that $\Psi$ is Lebesgue measurable for applying the previous proposition.
    For this, we have to show that $\forall (a, b) \times (c, d) \subset \R^2$, we have that:

    $$\Psi^{-1}((a, b) \times (c, d)) = \{x \in X: f(x) \in (a, b), g(x) \in (c, d)\} \in \M$$

    This can be done using the fact that $f$ and $g$ are Lebesgue measurable.

\end{proof}

\begin{fcorollary}
    Let $f, g: X \to \R$ be Lebesgue measurable. Then:
    \vspace{1em}
    \begin{itemize}
        \item $f + g$ is Lebesgue measurable
        \vspace{1em}
        \item $f \cdot g$ is Lebesgue measurable
    \end{itemize}

\end{fcorollary}

\vspace{1em}

\begin{fproposition}
    Let $(X, \M)$ be a measurable space (with $\M \supseteq \B(X)$), and
    $\{f_n\}_{n \in \N}$ be a sequence of Lebesgue measurable functions $f_n: X \to \R$.
    Then, the following functions are Lebesgue measurable:

    \vspace{1em}

    \begin{enumerate}
        \item $\sup_{n} f_n$
        \vspace{1em}
        \item $\inf_{n} f_n$
        \vspace{1em}
        \item $\limsup_{n} f_n$
        \vspace{1em}
        \item $\liminf_{n} f_n$
    \end{enumerate}

    In particular, if $\lim_{n} f_n$ exists, then it is Lebesgue measurable.

\end{fproposition}

\begin{proof}
    The proof goes as follows:

    \begin{enumerate}
        \item Since $\B(\R) = \sigma_0(\{(a, \infty): a \in \R\})$, it is enough to show that
        $\forall a \in \R$, we have that:

        $$(\sup_{n} f_n)^{-1}((a, \infty))= \{x \in X: \sup_{n} f_n(x) > a\} \in \M$$
    
        This can be done by using the fact that $f_n$ is Lebesgue measurable. 
        Indeed, we have that:
        
        $$\{x \in X: \sup_{n} f_n(x) > a\} = \bigcup_{n} \{x \in X: f_n(x) > a\} $$
        $$ = \bigcup_{n} f_n^{-1}((a, \infty)) \in \M$$

        because $f_n^{-1}((a, \infty)) \in \M$ for all $n$.
    
        \item The proof is analogous to the previous case, taking that:
        
        $$\inf_{n} f_n = -\sup_{n} (-f_n)$$

        \item We have that:
        
        $$\limsup_{n} f_n = \inf_{n} \sup_{k \geq n} f_k$$

        \item We have that:
        
        $$\liminf_{n} f_n = \sup_{n} \inf_{k \geq n} f_k$$

    \end{enumerate}
\end{proof}

\section{Properties holding almost everywhere}

\begin{fdefinition}
    Let $(X, \M, \mu)$ be a complete measure space. 
    We say that a property $P(x)$ holds \textbf{$\mu$-almost everywhere}
    (a.e) if:
    
    $$\mu(\{x \in X: P(x) \text{ is false}\}) = 0$$

    In other words, $P(x)$ holds $\mu$-almost everywhere if it holds everywhere except
    for a set of measure zero.

\end{fdefinition}

\begin{example}
    Let $f(x) = x^2$. Is it true that $f(x) > 0$ a.e.?\\

    We have that $\{x: x^2 \leq 0\} = \{0\}$
    \vspace{1em}

    \begin{itemize}
        \item In $(\R, \Lcur(\R), \lambda)$, the property is true a.e., since
        $\lambda(\{0\}) = 0$

        \item In $(\R, \power(\R), \mu_{\#})$ (counting measure), the property is false a.e.,
        since $\mu_{\#}(\{0\}) = 1$
    \end{itemize}
\end{example}

\begin{fproposition}
    Let $(X, \M, \mu)$ be a measure space:
    \vspace{1em}
    \begin{enumerate}
        \item $f: X \to \Rbar$ s.t. $f = g$ a.e, with $g$ measurable $\implies f$ is measurable
        \vspace{1em}
        \item $\{f_n\}_{n \in \N}$ a sequence of measurable functions s.t. $f_n \to f$ a.e., 
        then $f$ is measurable.
    \end{enumerate}
\end{fproposition}

\section{Simple functions}

\begin{fdefinition}
    Let $(X, \M)$ be a measurable space. A function $s: X \to \Rbar$ is 
    measurable and \textbf{simple} if $s$ is measurable and $s(X)$ is a finite set:

    $$s(X) = \{a_1, a_2, ..., a_k\}$$

    where $a_i \in \Rbar \; \forall i$, with $a_i \neq a_j$ for $i \neq j$.
    Then, $s$ can be written as:

    $$s(x) = \sum_{i=1}^{k} a_i \cdot \chi_{A_i}(x)$$

    where $A_i = s^{-1}(\{a_i\})$, $A_i \cap A_j = \emptyset$ for $i \neq j$,
    $\bigcup_{i=1}^{k} A_i = X$ and $A_i \in \M, \; \forall i$.
\end{fdefinition}

\textbf{\underline{Particular case:}}\\

If $X = \R$ (or $(a, b) \subset \R$) and $A_i$ is an interval $\forall i$, 
then $s$ is called a \textbf{step function}.\\

On the other hand, $\chi_{\Q}$ is a simple function, but not a step function:

$$\chi_{\Q}(x) = \begin{cases} 1 & x \in \Q \\ 0 & x \notin \Q \end{cases}$$

\begin{fremark}
    One may define simple functions without measurability requirements. 
\end{fremark}

\vspace{1em}

\textbf{\underline{Goal:}}\\
Approximate any measurable function $f: X \to \Rbar$ with (measurable and) 
simple functions.

\begin{ftheorem}[Simple approximation theorem (SAT)]

    Take $(X, \M)$ measurable space and $f: X \to [0, \infty]$, measurable.
    Then $\exists \{s_n\}_{n \in \N}$ a sequence of measurable, simple functions
    s.t. $s_1 \leq s_2 \leq ... \leq f$ pointwise (i.e., $\forall x \in X$) and:

    $$\lim_{n \to \infty} s_n(x) = f(x) \quad \forall x \in X$$

    Moreover, if $f$ is bounded, the convergence is uniform:

    $$\lim_{n \to \infty} \sup_{x \in X} |s_n(x) - f(x)| = 0$$
\end{ftheorem}

\begin{proof}
    In case $f$ is bounded, say $0 \leq f < 1$.\\ 
    
    For any $n \geq 1$, divide $[0, 1)$ into $2^{n}$ intervals of length 
    $2^{-n}$, and define:

    $$A_{n}^{(i)} = \{x \in X: \frac{i}{2^n} \leq f(x) < \frac{i + 1}{2^n}\}$$

    and:

    $$s_n(x) = \sum_{n = 0}^{2^n - 1} \frac{i}{2^n} \cdot \chi_{A_{n}^{(i)}}(x)$$

    One can show that such a sequence has the required properties
\end{proof}