\chapter{Spectral theory}

\begin{note}
    We will consider $E$ Banach, $T \in \Lcur(E, E) = \Lcur(E)$, and the problem:
    $$Tx = \lambda x \iff (T - \lambda I)x = 0$$
\end{note}

\begin{fdefinition}
    We define the following concepts:
    \vspace{1em}
    \begin{itemize}
        \item The \textbf{resolvent set} of $T$ is:
        $$\rho(T) = \{ \lambda \in \R: T - \lambda I: E \to E \text{ is bijective} \}$$

        \vspace{1em}
        \item The \textbf{spectrum} of $T$ is:
        $$\sigma(T) = \R \setminus \rho(T)$$

        \vspace{1em}
        \item $\lambda$ is an \textbf{eigenvalue} of $T$ if:
        $$Ker(T - \lambda I) \neq \{0\}$$

        where $Ker (T - \lambda I)$ is called the \textbf{eigenspace} corresponding to $\lambda$.
        Also:
        $$EV(T) = \{\text{eigenvalues of } T\} \subset \R$$
    \end{itemize}
\end{fdefinition}

\begin{fremark}
    Note that:
    $$EV(T) \subset \sigma(T)$$

    as $\lambda \in EV(T) \iff T - \lambda I$ is not injective. Also, note that if $dim E < \infty$, 
    then $EV(T) = \sigma(T)$. If $E$ has infinite dimension, then the inclusion may be strict.
\end{fremark}

\vspace{1em}

\begin{ftheorem}
    Let $E$ be Banach, $T \in \Lcur(E)$. Then:
    \vspace{1em}
    \begin{enumerate}[label=(\roman*)]
        \item $\sigma(T) \subset [- \norm{T}, \norm{T}]$
        \vspace{1em}
        \item $\sigma(T)$ is closed.
    \end{enumerate}
\end{ftheorem}

\begin{fremark}
    (i) means that the \say{spectral radius} is always $\leq$ the operatorial norm of $T$.
\end{fremark}

\begin{fremark}
    $|\lambda| > \norm{T} \implies T - \lambda I$ is invertible. Moreover, $\rho(T)$ is open.
\end{fremark}

\vspace{1em}

\begin{fexample}
    Let $E = \ell^2$. We define the \say{left shift operator} $T_{\ell}: \ell^2 \to \ell^2$ as
    follows:
    $$T_{\ell} x = (x^{(1)}, x^{(2)}, ...)$$
    
    for $x = (x^{(0)}, x^{(1)}, x^{(2)}, ...)$. Then, one can prove that $T_{\ell} \in \Lcur(\ell^2)$
    and $\norm{T_{\ell}} = 1$. By the theorem, $\sigma(T) \subset [-1, 1]$, closed.\\

    Also, notice that (for $\lambda = 0$) $T_{\ell}$ is surjective, but not injective. I.e., 
    $$R(T_{\ell}) = \ell^2, \quad Ker\; T_{\ell} = \{x \in \ell^2: x^{(k)} = 0 \; \forall k \geq 1, \; x^{(0)} \in \R\}$$

    Then, $\lambda = 0$ is an eigenvalue of multiplicity 1.\\

    Let us look for more eigenvalues: we know that $\lambda \in EV(T) \iff \exists x \neq 0$ s.t.:
    $$T_{\ell} x = \lambda x$$
    $$\iff (T_{\ell}x)^{(k)} = \lambda x^{(k)} \quad \forall k \geq 0$$
    $$\iff x^{(k + 1)} = \lambda x^{(k)} \quad \forall k \geq 0$$

    Take any $x^{(0)} = x_0 \neq 0$. Then, notice that:
    \begin{align*}
        x^{(1)} &= \lambda x_0\\
        x^{(2)} &= \lambda x^{(1)} = \lambda^2 x_0\\
        \vdots\\
        x^{(k)} &= \lambda x^{(k - 1)} = ... = \lambda^k x_0
    \end{align*}

    Then, $\lambda$ is an eigenvalue $\iff x = (x_0, \lambda x_0, \lambda^2 x_0, ...) = x_0 (1, \lambda, \lambda^2, ...) \in \ell^2$
    $$\iff \sum_{k=0}^{\infty} (\lambda^k)^2 < \infty \iff |\lambda| < 1$$

    Then, $EV(T) = (-1, 1)$ and
    $$(-1, 1) \subset \sigma(T) \subset [-1, 1]$$

    and because $\sigma(T)$ is closed, we conclude that $\sigma(T) = [-1, 1]$.
\end{fexample}

\begin{fexercise}
    Discuss $T_r: \ell^2 \to \ell^2$ the \say{right shift operator} such that:
    $$T_r x = (0, x^{(0)}, x^{(1)}, x^{(2)}, ...)$$

    Show that $T_r \in \Lcur(\ell^2)$ and $\norm{T_r} = 1$. Then, show that $\sigma(T_r) = [-1, 1]$
    and $EV(T_r) = \emptyset$.
\end{fexercise}

\section{Symmetric operators}

\begin{note}
    In what follows, consider:
    \begin{itemize}
        \item $(H, \intprod{\cdot}{\cdot})$ a Hilbert space.
        \item $T \in \K(X) = \K(X, X)$ a compact operator.
        \item $T$ is symmetric (self-adjoint)
    \end{itemize}
\end{note}

\begin{fdefinition}
    We say that $T$ is \textbf{symmetric} $\iff$
    $$\intprod{Tx}{y} = \intprod{x}{Ty} \quad \forall x, y \in H$$
\end{fdefinition}

\begin{fremark}
    If $T$ is symmetric, then:
    $$\norm{T}_{\Lcur(H)} = \sup_{x \neq 0} \frac{\intprod{Tx}{x}}{\norm{x}^2}$$

    This is called the \textbf{Rayleigh quotient}.
\end{fremark}

\vspace{1em}

\subsection{Fredholm's alternative theorem}

\begin{ftheorem}[Fredholm's alternative theorem]
    Let $H$ be Hilbert, $T \in \K(H)$ symmetric. Then:
    \vspace{1em}
    \begin{enumerate}[label=(\roman*)]
        \item $dim \; Ker (I - T) < \infty$
        \vspace{1em}
        \item $R(I - T)$ is closed.
        \vspace{1em}
        \item $Ker (I - T) = R(I - T)^{\perp}$ and $R(I - T) = Ker(I - T)^{\perp}$\\
        (in particular, $I - T$ is surjective $\iff$ is injective)
        \vspace{1em}
        \item Consider the following problem:
        $$(\star) = \begin{cases}
            \text{Given } f \in H, \text{ find } x \in H, \text{ s.t.:}\\
            (I - T)x = f
        \end{cases}$$

        Then, exactly one of the following is true:
        \vspace{1em}
        \begin{itemize}
            \item $\forall f, \; \exists ! x \in H$ solving $(\star)$
            \vspace{1em}
            \item $(\star)$ is solvable $\iff f \in Ker(I - T)^{\perp}$, and because
            $dim \; Ker(I-T) = N < \infty$, this means that:
            $$\intprod{f}{u_i} = 0 \quad \forall i = 1, ..., N$$

            s.t. $span(\{u_i\}_{i=1}^N) = Ker(I-T)$. 
        \end{itemize}
        
    \end{enumerate}
\end{ftheorem}

\begin{fremark}
    Consider $\lambda \neq 0$, $T - \lambda I$. Then, the FAT applies:
    $$T - \lambda I = - \lambda (I - \frac{1}{\lambda}T)$$

    where $\frac{1}{\lambda}T \in \K(H)$. As a consequence, we have that, for 
    $T \in \K(H)$ symmetric:
    $$\sigma(T) \setminus \{0\} = EV(T) \setminus \{0\}$$
\end{fremark}

\begin{fremark}
    For the theorem, there are some conditions that are not strictly necessary:
    \vspace{1em}
    \begin{itemize}
        \item $T$ symmetric is not necessary, as (i), (ii) are true, and (iii), (iv)
        can be formulated in terms of the adjoint operator $T^*$: $\intprod{Tx}{y} = \intprod{x}{T^*y}$.

        \vspace{1em}
        \item $H$ Hilbert is not necessary, for $E$ Banach, use duality pairing instead of 
        the scalar product.
    \end{itemize}
    \vspace{1em}
    Notice that, without the compactness assumption, the theorem breaks.
\end{fremark}

\section{Spectral theorem}

\begin{ftheorem}[Spectral theorem]
    Let $H$ be Hilbert and separable, with $dim H = \infty$. Let $T \in \K(H)$ symmetric.
    Then: 
    $$0 \in \sigma(T), \quad \sigma(T) \setminus \{0\} = EV(T) \setminus \{0\}$$

    and the following alternative holds:
    \vspace{1em}
    \begin{enumerate}[label=(\roman*)]
        \item Either $T$ has finitely many eigenvalues different from 0, and then
        $0 \in EV(T)$, with $dim \; Ker(T) = \infty$
        \vspace{1em}
        \item Or $EV(T) \setminus \{0\}$ is a sequence $\{\lambda_n\}_{n \in \N} \subset [-\norm{T}, \norm{T}]$
        and $\lambda_n \to 0$ as $n \to \infty$, i.e., $0 \in \sigma(T)$.
    \end{enumerate}
    \vspace{1em}
    Moreover, in both cases, there exists an orthonormal basis of $H$ made by the eigenvectors
    of $T$.
\end{ftheorem}

\begin{proof}
    The proof is based on the FAT, and it is omitted.

\end{proof}



