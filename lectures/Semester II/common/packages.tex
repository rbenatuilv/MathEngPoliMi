
\usepackage[utf8]{inputenc}
\usepackage[english]{babel}

% Miscelanea
\usepackage{amsmath,amssymb,amsbsy} %matemátias y símbolos
\usepackage{bbm}
\usepackage{hyperref} % hipervínculos en la tabla de contenidos
\usepackage{enumitem}
\usepackage{titlesec}  % Paquete para personalizar los títulos de secciones

%Cosas gráficas
\usepackage{graphicx}
\usepackage{tikz}
\usepackage{wrapfig}
\usepackage{pgfplots}
\usepackage{color}
\usepackage{booktabs}
\usepackage{tikz}
\usetikzlibrary{matrix}
\pgfplotsset{compat=1.17} 
\usepackage{float}
\usepackage{mathtools}
\DeclarePairedDelimiter\ceil{\lceil}{\rceil}
\usepackage{cancel}
\usepackage{xifthen}
\usepackage{longtable}


\usepackage{graphicx}
\usepackage{subcaption}
\usepackage{circuitikz}

\usepackage{amsthm} % Teoremas
\newtheorem{theorem}{Theorem}
\newtheorem{lemma}{Lemma}
\newtheorem{law}{Law}
% \newtheorem{corollary}{Corollary}
\newtheorem{definition}{Definition}
\newtheorem{proposition}{Proposition}
% \newtheorem{example}{Example}
\newtheorem{remark}{Remark}
% \newtheorem{exercise}{Exercise}

% Margenes
\usepackage{fullpage}

\usepackage{listings} % Paquete para incluir código fuente
\usepackage{xcolor}   % Paquete para definir colores (opcional, para personalización)

\usepackage{algorithm} % Paquete para escribir algoritmos
\usepackage{algpseudocode} % Paquete para escribir algoritmos

% Definir el estilo del código
\lstset{
  language=C++,               % Lenguaje de programación
  basicstyle=\ttfamily\footnotesize, % Estilo de la fuente del código
  keywordstyle=\color{blue},   % Color para palabras clave
  commentstyle=\color{gray},   % Color para comentarios
  stringstyle=\color{orange},  % Color para cadenas de texto
  numbers=left,                % Colocar números de línea a la izquierda
  numberstyle=\tiny\color{gray}, % Estilo de los números de línea
  stepnumber=1,                % Incremento de los números de línea
  frame=single,                % Coloca un marco alrededor del código
  breaklines=true,             % Divide líneas largas
  captionpos=b,                % Posición de la leyenda (abajo)
  showspaces=false,            % No mostrar espacios en blanco
  showstringspaces=false       % No mostrar espacios dentro de strings
}

\hypersetup{
	colorlinks,
	linkcolor={red!50!black},
	citecolor={blue!50!black},
	urlcolor={blue!80!black}
}

% Definir el estilo del capítulo usando tcolorbox
\titleformat{\chapter}[display]
  {\normalfont\huge\bfseries\centering} % Estilo de la fuente
  {\normalsize\chaptertitlename\ \thechapter} % Texto que aparece antes del título
  {1pt} % Espacio entre el título y el capítulo
  {\LARGE} % Estilo del título
  [\rule{\textwidth}{1pt}] % Línea horizontal después del título