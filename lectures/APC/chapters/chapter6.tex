\chapter{Inheritance and polymorphism}

\section{PIE properties}

Object Oriented Programming (OOP) is a programming paradigm that uses objects to design 
applications and computer programs. It is based on several principles, the most important 
of which are the PIE properties:

\begin{itemize}
    \item \textbf{Polymorphism}: The ability of an object to take on many forms. The most 
    common use of polymorphism in OOP occurs when a parent class reference is used to refer 
    to a child class object.
    \item \textbf{Inheritance}: The mechanism by which one class acquires the properties and 
    behavior of another class. It supports the concept of hierarchical classification.
    \item \textbf{Encapsulation}: The mechanism that binds together the code and the data it 
    manipulates, and keeps both safe from outside interference and misuse.
\end{itemize}

\section{Inheritance}

Inheritance is a mechanism in which one class acquires the properties and behavior of 
another class. It supports the concept of hierarchical classification. Inheritance is 
a powerful feature of OOP that allows the creation of a new class that is based on an
existing class. The new class inherits the attributes and methods of the existing class,
and can also add new attributes and methods of its own.\\

When a class inherits from another class, there are three main benefits. You can:

\begin{itemize}
    \item Reuse code: The methods and attributes of the parent class are inherited by the 
    child class, so you don't have to write them again.
    \item Extend functionality: You can add new methods and attributes to the child class.
    \item Override functionality: You can override the methods of the parent class in the 
    child class.
\end{itemize}

In general, inheritance establishes an "is-a" relationship between the parent class and
the child class. For example, if you have a class called \texttt{Animal}, you could create
a child class called \texttt{Dog} that inherits from \texttt{Animal}. In this case, you could
say that a \texttt{Dog} is an \texttt{Animal}.\\

A derived class inherits every data member of the base class, as well as every ordinary
member function. However, the derived class does not inherit the constructors, 
destructor, or assignment operators of the base class, as well as the friend functions
of the base class, since these are class-specific.\\

In C++, inheritance is specified by the colon (\texttt{:}) followed by the access specifier
(public, protected, or private) and the name of the base class. The access specifier
determines the visibility of the base class members in the derived class. The default
access specifier is private. In this course we will only use public inheritance.\\

The syntax for inheritance is as follows:

\begin{lstlisting}[language=C++]
class Base {
    // Base class members
};

class Derived : public Base {
    // Derived class members
};
\end{lstlisting}

\subsection{Protected members and class access}

In C++, the protected access specifier is similar to the private access specifier, but
with one key difference: protected members are accessible in the derived class. This
means that a derived class can access the protected members of its base class, but an
object (instance) of the derived class cannot.\\

The protected access specifier is useful when you want to make the data members of the
base class accessible to the derived class, but not to the outside world. This allows
you to encapsulate the data members of the base class and provide controlled access to
them through the derived class.\\

The syntax for protected members is as follows:

\begin{lstlisting}[language=C++]
class Base {
protected:
    // Protected members
};

class Derived : public Base {
    // Derived class members
};
\end{lstlisting}

Note that protected members are only accessible for the derived class through the
derived class object. They are not accessible through the base class object.

\subsection{Creation and destruction of derived classes}

When a derived class object is created, the following steps occur:

\begin{itemize}
    \item Space is allocated for the full object, that is, enouch space to store the data
    members of both the base and derived classes.
    \item The base class constructor is called to initialize the base class data members.
    \item The derived class constructor is called to initialize the derived class data members.
    \item The derived class object is created and is now usable.
\end{itemize}

When a derived class object is destroyed, the following steps occur:

\begin{itemize}
    \item The derived class destructor is called to destroy the derived class data members.
    \item The base class destructor is called to destroy the base class data members.
    \item The space allocated for the object is deallocated.
\end{itemize}

\subsection{Class design principle}

In the absence of inheritance, we can think of a class as having two different kinds
of developers: the class designer and the class user. The first group is responsible
for designing the class, while the second group is responsible for using the class.\\

When inheritance is used, there is a third group of developers: the class extender. This
group is responsible for extending the class by adding new attributes and methods to it.
The class extender is also responsible for overriding the methods of the base class.\\

When implementing inheritance, it is important to be strictest as possible with the
access specifiers. The base class should have all its members private, except for the
methods that are meant to be overridden by the derived class. These methods should be
protected. The derived class should have all its members private, except for the methods
that are meant to be used by the class user. These methods should be public.

\section{Polymorphism}

Polymorphism is the ability of objects to respond in different ways to the same message or
function call. An object has "multiple identities", based on its class inheritance tree, meaning
it can be use in different ways depending on the context.\\

There are two types of polymorphism: compile-time polymorphism and run-time polymorphism.\\

\textbf{Compile-time polymorphism} is achieved through function overloading and function
redefinition. Function overloading allows you to define multiple functions with the same
name but different parameter lists. Function redefinition allows you to define a function
with the same name and parameter list in a derived class as in the base class, to change
the behavior of the function.\\

\textbf{Run-time polymorphism} is achieved through virtual functions and inheritance. Virtual
functions are functions that are declared in the base class and can be overridden by the
derived class. When a virtual function is called through a base class pointer or reference,
the function that is called is determined by the type of the object, not the type of the
pointer or reference.\\

\subsection{Virtual functions}

In \texttt{C++}, a base class distinguishes functions that are type dependant from those that
it expects its derived classes to inherit withou modification. The former are declared as
\texttt{virtual} functions. A virtual function is a member function that is declared in the
base class using the keyword \texttt{virtual}. A virtual function can be overridden by a
derived class to provide a different implementation.\\

A derived class may or may not override a virtual function. If it does not override the
virtual function, the base class version of the function is used. If it does override the
virtual function, the derived class version of the function is used.\\

An important thing to notice is that virtual member functions support dynamic binding. This
means that the function bounds to the object at runtime, not at compile time. Without 
virtual member functions, \texttt{C++} uses static binding, which means that the function
bounds to the object at compile time, and it is only considered function redefinition.\\

The syntax for virtual functions is as follows:\\

\begin{lstlisting}[language=C++]
class Base {
public:
    virtual void function() {
        // Base class implementation
    }
};

class Derived : public Base {
public:
    void function() override {
        // Derived class implementation
    }
};
\end{lstlisting}

Through dynamic binding, we can use the same code to process objects of the base class and
objects of the derived class. This is a powerful feature of \texttt{C++} that allows us to
write more flexible and maintainable code. It is also a key feature of polymorphism.\\

Note that polymorphic behavior is only possible when using pointers or references to objects.
When using objects directly, the function that is called is determined by the type of the
object, not the type of the pointer or reference. Look at the following example:\\

\begin{lstlisting}[language=C++]
#include <iostream>

class Base {
public:
    virtual void function() {
        std::cout << "Base class implementation" << std::endl;
    }
};

class Derived : public Base {
public:
    void function() override {
        std::cout << "Derived class implementation" << std::endl;
    }
};

void f(Base& b) {
    b.function();
}

int main() {
    Base b;
    Derived d;

    f(b);
    f(d);

    return 0;
}

// Output:
// Base class implementation
// Derived class implementation
\end{lstlisting}

To summarize, a base class specifies that a member function should be dynamically bound by
declaring it as virtual. A derived class specifies that it intends to override a virtual
function by using the \texttt{override} keyword. Any non-static member function, other than
a constructor, can be declared as virtual.

\subsection{Derived-to-base conversion}

In \texttt{C++}, a derived class object can be assigned to a base class pointer or reference.
This is known as derived-to-base conversion. Derived-to-base conversion is useful when you
want to treat a derived class object as a base class object.\\

When a derived class object is assigned to a base class pointer or reference, only the base
class part of the object is accessible. This means that you can only access the base class
members of the object, not the derived class members.\\

For example:\\

\begin{lstlisting}[language=C++]
#include <iostream>

class Base {
public:
    virtual void function() {
        std::cout << "Base class implementation" << std::endl;
    }
};

class Derived : public Base {
public:
    void function() override {
        std::cout << "Derived class implementation" << std::endl;
    }
};

int main() {
    Derived d;
    Base* b = &d;

    b->function();

    return 0;
}

// Output:
// Derived class implementation
\end{lstlisting}

The fact that we can bind a reference or pointer to a base class object to a derived class
has an important implication: we don't know the actual type of the object to which the pointer
or reference is bound. This is a key feature of polymorphism. Tha object can be of the base
class type or of any derived class type.

\subsection{Static and dynamic types}

In \texttt{C++}, an object has two types: a static type and a dynamic type. The static type
of an expression is the type that is known at compile time, it is the the type with which
a variable is declared. The dynamic type of an expression is the type that is determined at
runtime, it is the type of the object to which a pointer or reference is bound.\\

When a base class pointer or reference is bound to a derived class object, the static type
of the pointer or reference is the base class type, and the dynamic type of the object is
the derived class type. This means that the compiler treats the object as if it were of the
base class type, but the object behaves as if it were of the derived class type.\\

It is important to notice that the dynamic type of an expression that is neither a pointer
nor a reference is always the static type.\\

In the previous example, the static type of the pointer \texttt{b} is \texttt{Base*}, and the
dynamic type of the object to which \texttt{b} is bound is \texttt{Derived}. This is why the
function that is called is the \texttt{Derived} class implementation.

\section{Abstract classes}

An abstract class is a class that cannot be instantiated, that is, you cannot create an object
of an abstract class. An abstract class is used as a base class for other classes, and it is
designed to be inherited by other classes. An abstract class is a class that contains one or
more pure virtual functions.

\subsection{Pure virtual functions}

A pure virtual function is a virtual function that is declared in the base class but has no
implementation. A pure virtual function is declared using the syntax \texttt{virtual void function() = 0;}.
A pure virtual function must be overridden by a derived class to provide an implementation.\\

For example:\\

\begin{lstlisting}[language=C++]
#include <iostream>

class Base {
public:
    virtual void function() = 0;
};

class Derived : public Base {
public:
    void function() override {
        std::cout << "Derived class implementation" << std::endl;
    }
};

int main() {
    Derived d;

    d.function();

    return 0;
}

// Output:
// Derived class implementation
\end{lstlisting}

In this example, the \texttt{Base} class is an abstract class because it contains a pure virtual
function. The \texttt{Derived} class is a concrete class because it provides an implementation
for the pure virtual function. The \texttt{Derived} class can be instantiated, but the \texttt{Base}
class cannot.

\subsection{Refactoring}

When refactoring a class, it is important to consider the following:

\begin{itemize}
    \item If a class is not meant to be instantiated, make it an abstract class by adding
    a pure virtual function.
    \item If a class is meant to be instantiated, make it a concrete class by providing
    an implementation for the pure virtual function.
    \item If a class is meant to be inherited by other classes, make it a base class by
    adding virtual functions.
    \item If a class is meant to be used by other classes, make it a utility class by
    adding static functions.
\end{itemize}

By following these guidelines, you can create a well-structured class hierarchy that is
easy to understand and maintain.