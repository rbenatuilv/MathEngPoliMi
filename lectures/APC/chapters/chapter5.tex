\chapter{Classes}

\section{What is a class?}

A class is a user-defined type. It specifies a blueprint for the creation of
objects. A class consists of a set of members:

\begin{itemize}
    \item Data members: variables that store the state of the object.
    \item Member functions (methods): functions that operate on the object.
\end{itemize}

The methods of a class can define the meaning of creation (constructor),
initialization, assignment, copy and cleanup (destructor), among other
operations, that determine the behavior of the objects of that class.

\section{Classes: \texttt{C++} general syntax}

In order to define a class in \texttt{C++}, we use the following syntax:\\

\begin{lstlisting}[language=C++]
class ClassName {
    public: // Accesible by all
        // functions
        // types
        // data (often best kept private!)

    private: // Accessible only by members of the class
        // functions
        // types
        // data
};
\end{lstlisting}

Notice that the members of a class can be public or private. Public members are
accessible from outside the class, while private members are only accessible
from within the class. A good practice is to keep the data members private and
provide public member functions to access and modify them. In that way, we can
control the access to the data and ensure that it is always in a valid state.\\

For example, we can define a class \texttt{Point} as follows:\\

\begin{lstlisting}[language=C++]
class Point {
    public:
        Point(int xx, int yy) : x(xx), y(yy) {} // Constructor
        int getX() { return x; }
        int getY() { return y; }
        void setX(int xx) { this->x = xx; }
        void setY(int yy) { this->y = yy; }
    private:
        int x;
        int y;
};
\end{lstlisting}

In this example, we define a class \texttt{Point} with two data members \texttt{x}
and \texttt{y} and four member functions: a constructor, two getter functions
\texttt{getX} and \texttt{getY}, and two setter functions \texttt{setX} and
\texttt{setY}.The members of a class can be accessed using the dot operator 
\texttt{.} for objects, and the arrow operator \texttt{->} for pointers to objects. 
Also, common operators such as \texttt{+}, \texttt{-}, \texttt{*}, \texttt{/}, 
among others, can be defined for a class.\\

Note that the public members of a class provide the class interface, while the private
members provide the class implementation. The class interface defines the
operations that can be performed on the objects of that class, while the class
implementation defines how those operations are performed.\\

Generally, we want to define the class interface, with all the public and private
member declarations, on a \texttt{.h} header file, while having the implementation of each 
function (including constructor and destructor) on a source \texttt{.cpp} file that 
includes this header. For example:

\begin{lstlisting}[language=C++]
// Point.h
#pragma once

class Point {
    public:
        Point(int xx, int yy); // Constructor
        int getX();
        int getY();
        void setX(int xx);
        void setY(int yy);
    private:
        int x;
        int y;
};
\end{lstlisting}

\begin{lstlisting}
// Point.cpp
#include "Point.h"

Point::Point(int xx, int yy) : x(xx), y(yy) {}
Point::getX() { return x; };
Point::getY() { return y; };
Point::setX(int xx) { x = xx };
Point::setY(int yy) { y = yy };
\end{lstlisting}

\section{Structs vs Classes}

In \texttt{C++}, the only difference between a \texttt{struct} and a \texttt{class}
is that the members of a \texttt{struct} are public by default, while the members
of a \texttt{class} are private by default. In general, we use \texttt{structs}
to define simple data structures, while we use \texttt{classes} to define more
complex data structures with associated operations.

\subsection{Structs in \texttt{C++}}

Structs are the simplest user-defined data structure that we have. As we said before, all its
members are public by default. This structure is inherited from the \texttt{C} language.
Its main syntax is:

\begin{lstlisting}[language=C++]
struct StructName {
    // data
    // Constructor
    // functions
};
\end{lstlisting}

For example, we can define a struct \texttt{Point} as follows:

\begin{lstlisting}[language=C++]
struct Point {
    int x;
    int y;
};
\end{lstlisting}

In general, we use structs to define simple data structures that group related
data together. Unlike classes, structs cannot define private members, so all
the date members of a struct are accessible from outside the struct.

\subsection{Public/private benefits}

The main benefit of using private members is that we can control the access to
the data and ensure that it is always in a valid state. For example, we can
define a class \texttt{Point} with private data members \texttt{x} and \texttt{y}
and provide public member functions to access and modify them. In that way, we
can ensure that the \texttt{x} and \texttt{y} coordinates of a point are always
non-negative.\\

In general, we use the public/private paradigm to:

\begin{itemize}
    \item Provide a clean interface to the users of the class.
    \item Hide the implementation details of the class.
    \item Allow the class to change its implementation without affecting its users.
    \item Easier to support code evolution.
    \item Mantain the class \textbf{invariants.}
\end{itemize}

\subsection{Invariants}

An invariant is a condition that must always be true for an object of a class.
It helps to ensure that the object is always in a valid state. For example, if we
define a class \texttt{Date} with data members \texttt{day}, \texttt{month} and
\texttt{year}, we can define, for example, the following invariants:

\begin{itemize}
    \item The day must be between 1 and 31.
    \item The month must be between 1 and 12.
    \item The year must be greater than 0.
\end{itemize}

We can enforce these invariants by defining the data members as private and
providing public member functions to access and modify them. In that way, we can
ensure that the \texttt{day}, \texttt{month} and \texttt{year} of a date are
always in a valid state.\\

In general, invariants help to ensure that the data that an object stores is always
correct and meaningful for its context. They help to prevent bugs and make the code 
easier to understand and maintain.\\

If we can't think of a good invariant for our data structure, we are probably dealing
with plain data, and if so, we may use a \texttt{struct} instead. But generally,
we should try to find good invariants for our user-defined types, so we can regularize
the behavior of our objects and prevent buggy code.

\section{\texttt{this} parameter}

When we call a member function, we do so on behalf of an specific object (an instance
of our class). Member functions access the object on which they were called through
an extra, implicit parameter named \texttt{this}. This parameter will be initialized
with the address of that object, so its data will be accesible from within the member
function.\\

For example, let us define a class that stores a date:\\

\begin{lstlisting}[language=C++]
class Date {
    public:
        Date(int dd, int mm, int yy) : d(dd), m(mm), y(yy) {}
        int day() { return d; }
        int month() { return m; }
        int year() { return y; }
    private:
        int d;
        int m;
        int y;
}
\end{lstlisting}

Then, when we call \texttt{object.month()} on a certain \texttt{object}, the compiler
automatically passes the address of that object to the method. Its if like the method 
were defined as:\\

\begin{lstlisting}[language=C++]
Date::month(Date *this) // Just a representation of what is happening
\end{lstlisting}

So, we have:

\begin{lstlisting}[language=C++]
Date my_birthday(26, 2, 2001);
int m = my_birthday.month()
// It is like we were writing Date::month(&my_birthday)
\end{lstlisting}

Note that inside the member functions, we are refering directly to the members of
the object on which the function was called, without using \texttt{this}.
Any direct use of a member of the class is assumed to be an implicit reference through
\texttt{this}. That is, when \texttt{month} uses \texttt{m}, it is as if we had
written \texttt{this -> m}.\\

It is then obvious that, when we define members methods of a class, it is forbidden to
use the keyword \texttt{this} for naming a parameter or a variable. Note that \texttt{this}
is a \texttt{const} pointer, meaning that we cannot change the address that it holds.

\section{\texttt{const} member functions}

When we define a member function of a class, we can specify that it is a \texttt{const}
member function by appending the \texttt{const} keyword to the function declaration.
A \texttt{const} member function is a member function that cannot modify the object
on which it was called.\\

For example, we can define a class \texttt{Date} with a \texttt{const} member function
\texttt{year} as follows:\\

\begin{lstlisting}[language=C++]
class Date {
    public:
        Date(int dd, int mm, int yy) : d(dd), m(mm), y(yy) {} // Constructor

        // ... Non-const member functions ...

        int year() const {
            ++y // (Imagine we do this by mistake)
            return y; 
        }
    private:
        int d;
        int m;
        int y;
}

Date my_birthday(26, 2, 2001);
int y = my_birthday.year(); // Will result in an error
\end{lstlisting}

In this example, we define a class \texttt{Date} with a \texttt{const} member function
\texttt{year} that tries to increment the year of the date. Since \texttt{year} is a
\texttt{const} member function, it cannot modify the object on which it was called.
Therefore, the statement \texttt{++y} will result in a compilation error.\\

In general, we use \texttt{const} member functions to ensure that the object on which
they were called is not modified. This helps to prevent bugs and make the code easier
to understand and maintain.\\

\section{Helper functions}

Helper functions are functions that are not members of a class, but that operate on
objects of that class. They are useful to define operations that are not part of the
class interface, but that are related to the class.\\

Usually, we want to keep the class interface clean and simple, and as minimal as
posible, so we define helper functions as non-member functions (outside the class) to
avoid cluttering the class interface, and when we need to define more complex operations.\\

For examplee, if we continue with the \texttt{Date} class, we can define a helper
function \texttt{next\_sunday()} that returns the next Sunday after a given date. We can
define this function as follows:\\

\begin{lstlisting}[language=C++]
class Date {
    // ... previous implementation ...
}

Date next_sunday(const Date &d) {
    // ... Implementation ...
    // returns a new Date object
}
\end{lstlisting}

Usually, we declare helper functions in the header file as the class, and define them
in the source file that includes the header. This way, we can keep the class interface
clean and simple, and provide the implementation details in the source file.

\section{Operator overloading}

In \texttt{C++}, we can define the behavior of operators for a class by overloading
them. Operator overloading allows us to define the meaning of operators such as
\texttt{+}, \texttt{-}, \texttt{*}, \texttt{/}, among others, for objects of a class.\\

When defining an operator for a class, we must define a member function or a non-member
function that specifies the behavior of that operator for objects of that class. The 
operator must have at leat one operand of the class for which it is defined.\\

Some advices for operator overloading are:

\begin{itemize}
    \item Define operators only when they make sense for the class.
    \item Define operators only with their conventional meaning.
    \item Don't overload operators like \texttt{*}, \texttt{\&\&}, \texttt{||}, \texttt{!}
\end{itemize}

\subsection{Operators as member functions}

When we define an operator as a member function, the object on which the operator
was called is the left operand of the operator, and the right operand is passed as
an argument to the member function.\\

For example, we can define the operator \texttt{+} for a class \texttt{Point} as a
member function as follows:\\

\begin{lstlisting}[language=C++]
class Point {
    public:
        // ... previous implementation ...

        Point operator+(const Point &p) {
            return Point(x + p.x, y + p.y);
        }

    private:
        // ... previous implementation ...
}
\end{lstlisting}

Note that the syntax for overloading an operator as a member function is to define
a member function with the name \texttt{operator} followed by the operator that we
want to overload. In this case, we define the operator \texttt{+} for the class
\texttt{Point} that adds two points.\\

We can also define operators between our class and other types. For example, when defining
the \texttt{[]} operator for a class, we can define it as a member function that takes
an integer as an argument. Let us see an implementation of this operator for a class
\texttt{Vector}:\\

\begin{lstlisting}[language=C++]
class Vector {
    public:
        // ... some implementation ...

        double &operator[](size_t i) const;

    private:
        std::vector<double> data;
}

double &Vector::operator[](size_t i) const { // Obs: returning a reference
    while (data.size() <= i) {
        data.push_back(0.);
    }
    return data[i];
}
\end{lstlisting}

In this example, we define the operator \texttt{[]} for the class \texttt{Vector} that
returns a reference to the element at the given index. If the index is out of bounds,
we push back zeros to the vector until the index is valid.\\

For operators as member functions, the left operand is always bounded to \texttt{this}.

\subsection{Operators as non-member functions}

When we define an operator as a non-member function, the object on which the operator
was called is passed as an argument to the function. This overloading is useful when
we want to define operators that are symmetric between two objects of the same class.\\

The syntax for overloading an operator as a non-member function is to define a non-member
function that takes two arguments of the class for which we want to overload the operator.\\

For example, we can define the operator \texttt{+} for a class \texttt{Point} as a
non-member function as follows:\\

\begin{lstlisting}[language=C++]
class Point {
        // ... some implementation ...
}

Point operator+(const Point &p1, const Point &p2) {
    return Point(p1.x + p2.x, p1.y + p2.y);
}
\end{lstlisting}

In this example, we define the operator \texttt{+} for the class \texttt{Point} as a
non-member function that adds two points. We have a problem here, because the operator 
\texttt{+} is not a member of the class \texttt{Point}, so it cannot access the private 
members of the class (in this case, \texttt{x} and \texttt{y}). To solve this, we can 
declare the operator \texttt{+} as a friend function of the class \texttt{Point}. 
It goes as follows:

\subsubsection{Friend functions}

When we define an operator as a non-member function, we can specify that the function
is a friend of the class. A friend function is a function that is not a member of the
class, but that has access to the private members of the class.\\

The syntax for defining a friend function is to declare the function as a friend of
the class in the class definition, and to define the function as a non-member function.\\

For example, we can define the operator \texttt{+} for a class \texttt{Point} as a
friend function as follows:\\

\begin{lstlisting}[language=C++]
class Point {
    public:
        // ... some implementation ...

        friend Point operator+(const Point &p1, const Point &p2);
}

Point operator+(const Point &p1, const Point &p2) {
    return Point(p1.x + p2.x, p1.y + p2.y);
}
\end{lstlisting}

In this example, we define the operator \texttt{+} for the class \texttt{Point} as a
friend function that adds two points. Now the operator \texttt{+} has access to the
private members of the class \texttt{Point}, so we can add two points without any
problem:\\

\begin{lstlisting}[language=C++]
    Point p1(1, 2);
    Point p2(3, 4);
    
    Point p3 = p1 + p2;
\end{lstlisting}

\subsection{Member vs non-member}

When we define an operator for a class, we can define it as a member function or as
a non-member function. The choice between the two depends on the context and the
semantics of the operator.\\

Here are some general guidelines for choosing between a member function and a non-member
function:

\begin{itemize}
    \item Must be member: \texttt{= [] () ->}
    \item Should be member:
        \begin{itemize}
            \item Compound assignments: \texttt{+= -= *= /=}, etc.
            \item Modify operator: \texttt{++ --}
        \end{itemize}
    \item Better non-member:
        \begin{itemize}
            \item Arithmetic operators: \texttt{+ - * /}
            \item Bitwise operators: \texttt{\& | \^}, etc.
            \item Comparison operators: \texttt{== != < >}, etc.
        \end{itemize}
    \item Better not overloaded: \texttt{* \&\& || !}
    \item Cannot be overloaded: \texttt{:: ?: .* .}
\end{itemize}

\subsection{Returning \texttt{this} object}

When we define an operator for a class, we can return the object on which the operator
was called by reference. This allows us to chain multiple operators together.\\

This is usually done when overloading the assignment operator \texttt{=} or the
compound assignment operators such as \texttt{+=}, \texttt{-=}, \texttt{*=}, \texttt{/=}.
For example, we can define the operator \texttt{+=} for a class \texttt{Point} as a
member function that returns the object on which the operator was called by reference:\\

\begin{lstlisting}[language=C++]
class Point {
    public:
        // ... some implementation ...

        Point &operator+=(const Point &p) {
            x += p.x;
            y += p.y;
            return *this;
        }
}
\end{lstlisting}

In this example, we define the operator \texttt{+=} for the class \texttt{Point} as a
member function that adds a point to another point and returns the object on which the
operator was called by reference. This allows us to chain multiple operators together:\\

\begin{lstlisting}[language=C++]
Point p1(1, 2);
Point p2(3, 4);
Point p3(5, 6);

p1 += p2 += p3;
\end{lstlisting}

